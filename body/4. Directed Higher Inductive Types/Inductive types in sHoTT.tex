\documentclass[main.tex]{subfiles}

\begin{document}
Inductive types, are intuitvely viewed as types which are completely "determined" by their constructors, of which map into the inductive type
being formed as \textit{points}. They are positive types which contain no higher homotopical data.
We would similarly like to state the same for sHoTT, that inductive types are objects without any higher \textit{simplicial data}. We observe that this is
trivially the case for the Unit type. Since all of its terms are definitionally $\star$, the morphism space of $\Un$ is
\begin{equation}
    \label{eq:unhom}
    \ndexten{\osx}{\Un}{\pardo}{[\star,\star]}
\end{equation}
which clearly has contractible fibers. Since we've assumed relative functional extensionality, the entire type (\ref{eq:unhom}) will be contractible.

As a less degenerate task, we try the same thing for the natural numbers. In service of that goal, we can define a type family $$\co : \mathbb{N} \to \mathbb{N}\to \U $$ recursively, with defining equations
\begin{align*}
    &\co\,0\,0\defeq \Un&\\
    &\co\,0\,S(n) \defeq \vo&\\
    &\co\,S(n)\,0\defeq \vo&\\
    &\co\,S(n)\,S(m)\defeq \co \,n\,m&
\end{align*}

Before we can define $\eco$ and $\dco$, we define a map $\text{pred} : \mathbb{N} \to \mathbb{N}$ as
\begin{align*}
    &\text{pred}\,0 \defeq 0&\\
    &\text{pred}\,S(n) \defeq n&
\end{align*}
Moreover, there is a map $\phi : \prd{n:\mathbb{N}}{m:\mathbb{N}}\ho[\mathbb{N}]{S(n)}{S(m)} \to \ho[\mathbb{N}]{n}{m}$ defined as
\begin{align*}
    &\phi \, 0 \, 0 \,\_ \defeq \idhom{0}&\\
    &\phi \, S(n)\, S(m)\,f \defeq \lam{i}\text{pred}(f(i))
\end{align*}

Now, we define $\eco : \prd{n : \mathbb{N}}{m : \mathbb{N}} \ho[\mathbb{N}]{n}{m} \to \co\,n\,m$ with the defining equations
\begin{align*}
    &\eco\,0\,0\,\_ \defeq \star&\\
    &\eco\,S(n)\,S(m)\,f\defeq \eco\,n\,m\,(\phi\,n\,m\,f)&
\end{align*}

and we define $\dco : \prd{n:\mathbb{N}}{m:\mathbb{N}}\co\,n\,m \to \ho[\mathbb{N}]{n}{m}$ as
\begin{align*}
    &\dco\,0\,0\,\_ \defeq \idhom{0}&\\
    &\dco\,S(n)\,S(m)\,u\defeq \lam{i}S((\dco\, n\, m\, u)i)&
\end{align*}
We can also show that the fibers of $\co$ are propositional by defining a type $\psi : \prd{n : \mathbb{N}}{m : \mathbb{N}}\co\,n\,m \to (\co\,n\,m = \Un)$ by
\begin{align*}
    &\psi\,0\,0\,\_\defeq \refl{\Un}\\
    &\psi\,S(n)\,S(m)\,u\defeq\psi\,n\,m\,u&
\end{align*}

Finally, we can show that these functions form a bi invertible equivalence by defining a map $$\epsilon : \prd{n:\mathbb{N}}{m:\mathbb{N}}{u : \co \, n\, m}\eco\,n\,m\,(\dco\,n\,m \,u) =u$$
as
\begin{align*}
    \epsilon\,n\,m\,u \defeq \text{ap}_{((\psi\,n\,m\,u)^{-1})_*}(\refl{\star})
\end{align*}
Noting that 
\begin{equation}
    \eco\,n\,m\,(\dco\,n\,m \,u) =_{\co\,n\,m} u \simeq \star =_{\Un} \star
\end{equation}
since $\text{ap}_{(\psi\,n\,m\,u)_*}: \eco\,n\,m\,(\dco\,n\,m \,u) =_{\co\,n\,m} u \to \star =_{\Un} \star$ is an equivalence with inverse as $\text{ap}_{((\psi\,n\,m\,u)^{-1})_*}$ as transporting 
is an equivalence.

On the other hand, if we try to define $$\eta : \prd{n:\mathbb{N}}{m:\mathbb{N}}{f : \ho[\mathbb{N}]{n}{m}}\dco\,n\,m\,(\eco\,n\,m f) =f,$$
we run into issues as we have no way of showing that, in the base case, that all $f : \ho[\mathbb{N}]{0}{0}$ have the property that $$f = \refl{0}.$$
In fact, we will ultimately have to inhabit the type $$\prd{n : \mathbb{N}}{m : \mathbb{N}}\ho[\mathbb{N}]{n}{m} \to \ho[\mathbb{N}]{n}{m}=\Un.$$
That is, we want to show that the two-sided representable for the naturals also has propositional fibers. From there, we can easily complete
$\eta$. To do so, though, is equivalent to a proof that $\mathbb{N}$ is Segal, which ultimately requires us to have a characterization of the 
morphism types in the naturals.
\begin{lemma}
    $\prd{n : \mathbb{N}}{m : \mathbb{N}}\ho[\mathbb{N}]{n}{m} \to \ho[\mathbb{N}]{n}{m}=\Un$ if and only if $\mathbb{N}$ is Segal.
\end{lemma}
\begin{proof}
    In the forward direction, we assume there is a term 
    $$\zeta : \prd{n : \mathbb{N}}{m : \mathbb{N}}\ho[\mathbb{N}]{n}{m} \to \ho[\mathbb{N}]{n}{m}=\Un.$$
    Then, we can complete the definition of $\eta$ very quickly:
    \begin{equation}
        \eta\,n\,m\,f \defeq \text{ap}_{((\zeta\,n\,m\,f)^{-1})_*}(\refl{\star})
    \end{equation}
    So, we conclude that for all $n,m : \mathbb{N}$,
    $$\ho[\mathbb{N}]{n}{m} \simeq \co\,n\,m.$$
    The astute reader will notice that $\co$ defined above is exactly $\co$ defined in theorem 2.13.1. Hence, $$(n=m)\simeq \co(n,m)\simeq\ho[\mathbb{N}]{n}{m}.$$
    That is, $\mathbb{N}$ is discrete, so it is Segal. 
    
    On the other hand, given a proof that $\mathbb{N}$ is Segal, we can characterize the morphism space in a way that looks
    much more similar to book HoTT. Define a map $$\co' : \mathbb{N}\to\mathbb{N}\to\uc$$
    as
    \begin{align*}
        &\co'\,0\,0\defeq \Un&\\
        &\co'\,0\,S(n) \defeq \vo&\\
        &\co'\,S(n)\,0\defeq \vo&\\
        &\co'\,S(n)\,S(m)\defeq \co' \,n\,m&
    \end{align*}
    There is also a dependent function $ r : \prd{n : \mathbb{N}} \co' \,n\,n$ defined as 
    \begin{align*}
        r\,0 \defeq \star
        r\,S(n) \defeq r(n)
    \end{align*}
    We can define $\eco' : \prd{n : \mathbb{N}}{m : \mathbb{N}} \ho[\mathbb{N}]{n}{m} \to \co'\,n\,m$ as
    \begin{align*}
        \eco'\,n\defeq \text{yon}^{\lam{m}\co'\,n\,m}_{n}(r(n))
    \end{align*}
    and $\dco' : \prd{n:\mathbb{N}}{m:\mathbb{N}}\co'\,n\,m \to \ho[\mathbb{N}]{n}{m}$ as
    \begin{align*}
        &\dco\,0\,0\,\_ \defeq \idhom{0}&\\
        &\dco\,S(n)\,S(m)\,u\defeq \lam{i}S((\dco\, n\, m\, u)i)&
    \end{align*}
    We must now prove that $\eco'$ and $\dco'$ are quasi-inverses. We do so by defining a map 
    $$\eta' : \prd{n:\mathbb{N}}{m:\mathbb{N}}{f : \ho[\mathbb{N}]{n}{m}}\dco'\,n\,m\,(\eco'\,n\,m f) =f$$
    as
    \begin{equation}
        \eta' \, n \defeq \text{yon}^{\lam{(m,f)}\dco'\,n\,m\,(\eco'\,n\,m f) =f}_n(\refl{\idhom{0}})
    \end{equation}
    with this being a valid definition as $\lam{(m,f)}\dco'\,n\,m\,(\eco'\,n\,m f) =f$ \todo{does this follow?} is covariant due to the
    covariance of $\lam{m}\ho[\mathbb{N}]{n}{m}$, which follows from the assumption that $\mathbb{N}$ is Segal.

    To define the final term showing that $\eco'$ and $\dco'$ are quasi-inverses, we note that we can perform a similar
    meanuvar as in the non-Segal case and define a map showing that the fibers of $\co'$ are mere propositions. So, we define a map
     $\psi' : \prd{n : \mathbb{N}}{m : \mathbb{N}}\co\,n\,m \to (\co\,n\,m = \Un)$ as
    \begin{align*}
        &\psi\,0\,0\,\_\defeq \refl{\Un}\\
        &\psi\,S(n)\,S(m)\,u\defeq\psi\,n\,m\,u&
    \end{align*}
    and define 
$$\epsilon' : \prd{n:\mathbb{N}}{m:\mathbb{N}}{u : \co' \, n\, m}\eco'\,n\,m\,(\dco'\,n\,m \,u) =u$$
as
\begin{align*}
    \epsilon\,n\,m\,u \defeq \text{ap}_{((\psi'\,n\,m\,u)^{-1})_*}(\refl{\star})
\end{align*}
    Thus, for all $n,m : \mathbb{N}$ it is the case that
    $$\ho[\mathbb{N}]{n}{m}\simeq \co'\,n\,m,$$ hence $\psi'$ induces a map
    $$\prd{n : \mathbb{N}}{m : \mathbb{N}}\ho[\mathbb{N}]{n}{m} \to (\ho[\mathbb{N}]{n}{m} = \Un)$$
\end{proof}

So, out of the box, we quickly see that we cannot immediately conclude that inductive types behave as expected. It is only with
the added assumption of Segalness that the type $\mathbb{N}$ satisfied the properties we would expect from an inductive type.
This goes well with the slogan of sHoTT: "\textit{some} types are $(\infty,1)$-categories". 

\end{document}