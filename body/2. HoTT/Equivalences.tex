\documentclass[main.tex]{subfiles}

\begin{document}
In HoTT, for a map $f : A \to B$, the notion of an inverse, which we denote its \textbf{quasi-inverse} and represent the type of all such
quasi-inverses with the type
\begin{equation}
    \qinv(f) \defeq \sm{g : B \to A}((f\circ g \sim \id{B})\times(g \circ f \sim \id{A})),
\end{equation} is not the premiere notion of equivalence. Instead, we can split the inverse into two parts. The first we denote
as a \textbf{left inverse} and write the type of all left inverses as 
\begin{equation}
    \linv(f) \defeq \sm{g : B \to A}g \circ f \sim \id{A}.
\end{equation}
The second is denoted as a \textbf{right inverse} and the type of all right inverses is written as
\begin{equation}
    \rinv(f) \defeq \sm{g : B \to A} f \circ g \sim \id{B}.
\end{equation}
A map $f$ is a \textbf{bi-invertible} map if it has both a left and right inverse. That is, if the type
\begin{equation}
\biinv(f) \defeq \linv(f) \times \rinv(f)
\end{equation}
is inhabited. Bi-invertible maps are the premiere notion of equivalence for this text. That is,
\begin{equation}
    \iseq \defeq \biinv
\end{equation}

If $f$ is an equivalence, we also call it an \textbf{isomorphism}. We can recover the set theoretic
notions of surjection and injection as well.
\begin{definition}
    For any $f : A \to B$
    \begin{enumerate}
        \item If for every $b:B$ the type $\trunc{}{\hfib{f}{b}}$ is inhabited, $f$ is said to be \textbf{surjective}
        \item If for every $x,y : A$ the function $\apfunc{f}:(x=_A y) \to (f(x) =_B f(y))$ is an equivalence, $f$ is said to be an
            \textbf{embedding}.
    \end{enumerate}
\end{definition}

We can prove that the two notions coincide.

\begin{lemma}
    \label{lem:equivissurjemb}
    A function $f : A \to B$ is an equivalence if and only if it is both surjective and an embedding.
\end{lemma}

Given two type families $P,Q : A \to \U$, we say a map $f : \prd{x:A}P(x) \to Q(x)$ is a \textbf{fiberwise map}. Fiberwise maps induce maps
between total spaces.
\begin{definition}[HoTT Book 4.7.5]
    For any type families $P,Q : A \to \U$ and fiberwise map $f : \prd{x:A}P(x) \to Q(x)$, 
    \begin{equation}
        \tot(f) \defeq \lam{(x,u)}(x, f(x,u)) : \sm{x : A}P(x) \to \sm{x:A}Q(x)
    \end{equation}
\end{definition}

We can prove closure properties as well. We say that $f$ is a fiberwise equivalence if, for each $f(x) : P(x) \to Q(X)$ is an equivalence.

\begin{lemma}[HoTT Book 4.7.7]
    \label{lem:equivistotequiv}
    For any type families $P,Q : A \to \U$ and fiberwise map $f : \prd{x:A}P(x)\to Q(x)$, $f$ is a fiberwise equivalence if and only if
    $\tot(f)$ is an equivalence.
\end{lemma}
We can "transport" structure along equivalences as well to change base types in a total space.
\begin{lemma}[HoTT Book Exercise 2.17]
    \label{lem:equivbaseequivtot}
    If $e : A \to B$ is an equivalence with quasi-inverse $e^{-1}$, 
    \begin{equation}
        \sm{a : A}B(a) \simeq \sm{b:B}B(e^{-1}(b))
    \end{equation}
\end{lemma}

Path composition is also an equivalence.
\begin{lemma}[HoTT Book Exercise 2.6]
    \label{lem:compisequiv}
    For any $p : x = y$, the function
    \begin{equation}
        (p \sq -) : (y = z) \to (x = z)
    \end{equation}
    is an equivalence.
\end{lemma}

\end{document}