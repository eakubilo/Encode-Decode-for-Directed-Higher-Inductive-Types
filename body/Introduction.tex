\documentclass[main.tex]{subfiles}
\begin{document}
Martin-Löf Type Theory (MLTT), alternatively referred to as Intuitionistic Type Theory, is a constructivist formal logic system that presents a distinct foundation compared to set theory. A key difference lies in its primitive notion of a collection: instead of \textit{sets}, MLTT utilizes \textit{types}. These types, unlike sets, operate under different principles. Crucially, MLTT leverages the Curry-Howard isomorphism to establish a connection between mathematical logic and programming languages. This isomorphism manifests in the ``proofs as programs'' paradigm, where \textit{type specifications} are understood as \textit{propositions} of formal logic. Consequently, the typing judgement
\[ a : A \]
is also read as
\[
a \text{ is a proof of the proposition } A.
\]
This connection between types and propositions enables \textit{proof-relevant mathematics}, a framework for manipulating concrete ``proof objects" not found in classical mathematics. The analogy doesn't stop there. A further essential element of type theory is the \textit{type-indexed family of types}, often shortened to \textit{family} or \textit{type family}, whose typing judgement is expressed as:
\[
x : A \vdash B (x) : \U.
\]
Type families can be interpreted as representing a \textit{hypothetical judgement}, asserting that $B(x)$ is a proposition under the assumption $x:A$.

Another fundamental distinction between MLTT and classical mathematics lies in the nature of equality. In MLTT, equality manifests in two distinct forms. The first pertains to computation. We write \[ a \equiv b : A\] and say that $a$ \textbf{is definitionally equal to} $b$. This form of equality is inherent and holds \textit{a priori} by definition, requiring no evidence. Crucially, definitional equality allows for direct substitution: if $a \equiv b$, then we immediately know that $B(a) \equiv B(b)$.

The second form of equality addresses \textit{provability}. In many situations, particularly in mathematics, we desire to consider objects with the same fundamental properties as being equal, even if they are not definitionally identical. A prime example is category theory, where equality extends beyond simple definition. It is standard practice to treat \textit{equivalent} objects as equal, a concept reflected in the fact that most categorical definitions hold ``up to canonical isomorphism," or \textit{equivalence}. Consider, for instance, sets $C$ and $D$ within the category of sets ($\mathsf{Set}$), and the Cartesian product operation $\times$. While technically distinct, the product sets $C \times D$ and $D \times C$ are ``functionally" the same, both satisfying the universal property of the product. Similarly, one might wish to consider the sets $C \times \mathbbm{1}$, $\mathbbm{1} \times C$, and $C$ as representing the same underlying object.

However, attempting to directly impose a \textit{monoid} structure on the category of sets based on strict equality immediately encounters a problem: the sets $C$ and $C \times \mathbbm{1}$ are not, in fact, equal. To navigate these situations where strict equality is too restrictive, we develop more flexible notions of ``sameness" and construct our definitions accordingly. As previously mentioned, in category theory, the concept of \textit{equivalence} precisely captures this desired notion of sameness. Therefore, categories that satisfy the group axioms not by strict equality but ``up to equivalence" are termed \textit{symmetric monoidal categories}, and ($\mathsf{Set}$) provides a fundamental example of such a structure.

As a built-in feature, type theory offers a solution for representing these more flexible notions of ``sameness" through \textit{identity types}. Specifically, for any two terms within a type, type theory provides a type whose elements are precisely the proofs of equality between those terms. We write the type of proofs witnessing the identity of two terms as \[x=_A y : \U,\] interpret this as ``''$x$ \textbf{is propositionally equal to} $y$", and call an inhabitant $p : x =_A y$ a \textbf{path} from $x$ to $y$. These identity types truly embody the notion of equality, as evidenced by the definability of terms demonstrating that the identity type satisfies the axioms of an equivalence relation:
 $$\refl{\text-} : \prd{x:A}a=_A a,$$
 $$(\text-)^{-1}:\prd{x,y:A}(x =_A y ) \to (y =_A x),$$ and
 $$\text-\circ\text- : \prd{x,y,z : A} (x =_A y) \to (y =_A z) \to (x =_A z).$$
These terms form data witnessing that identity types are reflexive, symmetric, and transitive respectively.

One of the built-in strengths of our type theory is Leibniz's Law, also known as the indiscernibility of identicals. This means that given a proof of equality $p : x =_A y$, we can naturally transform elements of the type family $B(x)$ into elements of $B(y)$. This transformation, known as a coercion, is not just a simple function; it is a full equivalence, $B(x) \simeq B(y)$.

Type theory also exhibits more explicit connections with category theory. Notably, Hofmann and Streicher's observation \cite{hofmann1998groupoid} revealed that MLTT types possess the structure of groupoids – categories where every morphism is an isomorphism. Under this interpretation, a $p : x=_A y$ (and any path propositionally equal to $p$) corresponds to a morphism $f : \ho[A]{x}{y}$\footnote{This is meant categorically, not type theoretically.}. Essentially, morphisms are equivalence classes of paths, where the equivalence relation is propositional equality. Furthermore, identity types satisfy coherence rules like unit laws and associativity, ensuring that this interpretation of types as groupoids within a category is well-founded. However, this groupoid interpretation doesn't fully capture the richness of type theory. For instance, given three paths $p : x =_A y$, $q : y=_A z$, and $r : z =_A w$, there is an identification
 \[
 (p \cdot q) \cdot r =_\U p \cdot (q \cdot r).
 \]
While these two morphisms are not definitionally equal, they are identified as the same morphism due to our quotienting operation under propositional equality. This discrepancy strongly suggests the presence of a richer, higher-dimensional structure within types. Indeed, this observation motivates the interpretation of types as weak $\infty$-groupoids \cite{berg_garner_richard, AWODEY_2009, Gambino_2008, kapulkin2018simplicialmodelunivalentfoundations, Warren2008HomotopyTA}, a more sophisticated notion than simple groupoids. An $\infty$-groupoid can be informally understood as a category where morphisms possess a dimension, with higher-level morphisms constructed from lower ones, and crucially, all morphisms at every level are invertible. To illustrate this dimension property, we can borrow topological language: objects are $0$-cells, morphisms between objects are $1$-cells, morphisms between $1$-cells are $2$-cells, and so on, ad infinitum.

Beyond the inherent structure of types, two fundamental questions remain:
\begin{enumerate}
  \item Under what conditions does an equivalence of types, $B(x) \simeq B(y)$, imply an equality of types, $B(x)=_{\U} B(y)$?
  \item How can we formally define basic \textit{free} $\infty$-groupoids arising in topology, such as the circle, torus, and cell complexes in general?
\end{enumerate}

While MLTT lacks the expressive power to generally answer these questions, extending it with Voevodsky's univalence axiom and higher inductive types provides a positive resolution. This extended system, known as Homotopy Type Theory (HoTT), directly addresses these limitations. Voevodsky's univalence axiom formalizes the principle that ``equivalent structures are equal." Consequently, we can assert that type families indeed act functorially with respect to propositional equality. Specifically, univalence allows us to conclude that an equivalence $B(x) \simeq B(y)$ implies an equality $B(x) = B(y)$. Furthermore, higher inductive types enhance the inductive type schema by allowing for \textit{path} constructors, not just point constructors. This provides a logical framework for defining common topological spaces like the circle and torus mentioned earlier, as well as more complex constructions such as homotopy pushouts and colimits.

The introduction of univalence and higher inductive types to MLTT, forming HoTT, is specifically intended to guide the interpretation of types as reflecting the homotopy structure of topological spaces. This foundational choice naturally leads to a homotopical interpretation of the type theory. A central concept within homotopy theory is that of a \textit{fibration}. In essence, a fibration $p : E \to B$ is a special kind of map between spaces characterized by certain ``lifting" properties, crucially the ability to lift paths in the base space $B$ to corresponding paths in the total space $E$. Therefore, in HoTT, we consider type families $P : A \to \U$ as the type-theoretic counterparts of \textit{fibrations}. This perspective is grounded in the fact that type families, like fibrations, induce maps from their total space $\sm{x:A}P(x)$ down to the base space $A$, and characteristically exhibit the path lifting property.

All of this machinery culminates in the view that HoTT provides a language for doing homotopy theory synthetically, without building up point-set topology first. To illustrate this synthetic approach, consider the calculation of the fundamental group of the circle. In algebraic topology, this calculation is a significant undertaking, requiring the development of homotopy lifting properties and the theory of covering spaces, all within the framework of point-set topology. The traditional method hinges on demonstrating that the winding map  $w : \mathbb{R} \to S^1$ is a fibration, a property which allows one to relate loops in the circle to their ``windings" in the real line. The proof of this fibration property and the subsequent calculations often involve meticulous manipulation of open sets, limits, and other point-set constructions.

Alternatively, in type theory, the calculation is strikingly direct \cite{licata_shulman_fundamental}. After defining the circle $S^1$ as a higher inductive type with a basepoint $\ba : S^1$ and a path $\mathsf{loop} : \ba = \ba$, we can define a type family $\co : S^1 \to \U$ by specifying its values on the constructors of the circle:
\begin{align*}
	&\co(\ba) \defeq \mathbb{Z} \\
	&\co(\mathsf{loop}) \defeq \mathsf{ua}(\mathsf{succ})
\end{align*}
Here, $\mathbb{Z}$ represents the integers, $\mathsf{succ}$ is the successor function on the integers, and $\mathsf{ua}$ (from the univalence axiom) lifts this function to an equivalence between types. Crucially, in HoTT, any function between types can be viewed as a fibration, so we get the structure of a fibration ``for free". From this, we construct maps $\eco : \prd{x:S^1}(\ba = x) \to \co(x)$ and $\dco : \prd{x : S^1}\co(x) \to (\ba = x)$ that witness the equivalence between the loop space of the basepoint and the fiber of the type family at the basepoint.

For $\eco$, given a point $x:S^1$ and a path $p : \ba = x$, we define $\eco(x,p) \defeq \mathsf{transport}^\co(p,0)$. Here, $\mathsf{transport}^\co(p,-)$ describes how elements of $\co(\ba)$ are transported along the path $p$ to elements of $\co(x)$. Since $0 : \co(\ba)$, applying $\mathsf{transport}^\co(p,-)$ to $0$ yields an element of $\co(x)$. Intuitively, $\eco$ translates a loop based at $\ba$ into an integer representing its winding number.

Conversely, $\dco$ maps elements of $\co(x)$ back to paths based at $\ba$. While its definition involves pattern matching on the constructors of the circle, the core idea is to map integers to corresponding loops in the circle.

Showing that $\eco$ and $\dco$ are inverses requires the induction principle of $S^1$ and the properties of transport, as well as the type of equational reasoning about paths that is central to HoTT. Finally, the fundamental group of the circle is calculated by observing that the loop space at the basepoint is equivalent to the fiber of the type family at the basepoint, which is equivalent to the integers: $\ba = \ba \simeq \co(\ba) \equiv \mathbb{Z}$. This demonstrates that the fundamental group of the circle is indeed the integers, a result derived directly and synthetically within the framework of HoTT.


HoTT, despite its strengths, faces challenges when dealing with asymmetric relationships. To illustrate this concretely, we draw upon examples from \cite{daniel_r_licata_2-dimensional_2011, gratzer_directed_2024} and consider the formalization of an inductive family we'll call $\Li$. As a starting point, we define a monoid as a type $A$ together with a binary operation $\star : A \times A \to A$ satisfying associativity and having an identity element. This can be formally expressed as:

\[
\mon \defeq \sm{A : \U}{\star : A \times A \to A}\assoc(A, \star) \times \iden(A, \star)
\]
with $\assoc$ defined as
\[
\assoc \defeq \prd{A : \U}{\star : A \times A \to A}{a,b,c : A} a \star (b \star c) = (a \star b) \star c
\]
and $\iden$ defined as
\[
\iden \defeq \prd{A : \U}{\star: A \times A \to A}\sm{e : A}\prd{a : A} (e \star a = a) \times ( a \star e = a).
\]
As the names imply, $\assoc$ and $\iden$ express that its input satisfies associativity and has an identity, respectively. Having established the definition of a monoid, we now introduce, for each monoid $A : \mon$, an inductive type $\Li\,A$ generated by:
\begin{itemize}
  \item $\nil : \Li\,A$
  \item $\cons{a}{l} : \Li{A}$, for every element $l : \Li\,A$ and $a : \fst(A)$.
\end{itemize}
We then define a function $\add : \prd{A : \mon}\Li{A} \to A$ using $\Li$ recursion. With this setup, we can ask: what structures does $\add$ preserve? Demonstrating that $\add$ preserves equality is straightforward; for instance, if $A = B$, then $\add\,A = \add\,B$. However, our mathematical intuition leads us to conjecture a broader claim: that $\add$ also preserves preorders, which are characterized by reflexivity and transitivity.

To attempt to prove our conjecture within Homotopy Type Theory (HoTT), we might initially consider the types $\mon$ and $\Li{A}$ as categories, as defined in \cite[9]{program_homotopy_2013}. However, categories in HoTT are not a primitive notion but rather an analytic construction. Therefore, to even work with the category $\mon$, the type of morphisms $\ho[\mon]{A}{B}$ must be explicitly defined and shown to satisfy the category axioms. Assuming we can establish this, and given a monoid morphism $f : \ho[\mon]{A}{B}$, we then encounter the issue that the type of morphisms between the lifted structures, $\ho{\Li{A}}{\Li{B}}$, is not automatically inhabited. Consequently, proving that $\Li$ acts as a functor, as defined in *op. cit.*, becomes a necessary step. Moreover, while we might hope to leverage the parametricity of $\add$ to obtain a proof of our conjecture as a ``theorem for free" \cite{wadler1989theorems}, articulating this theorem requires first defining a map $f_* : \Li{A} \to \Li{B}$.

Once such a map $f_*$ is defined, the corresponding ``free theorem" would ideally be an element of the type $\add{B} \circ f_* = f \circ \add{A}$. However, despite the theoretical possibility of these constructions, the inherent focus on symmetric relations within our type theory prevents a natural and direct handling of asymmetric relations like preorders.

To address this limitation, \textit{directed type theory} has been investigated. These type theories generally fall into two primary variations. The first approach operates from the perspective that \textit{every} type corresponds to an $(\infty,1)$-\textit{category}\footnote{An $(\infty,1)$-category is a category with an infinite hierarchy of morphisms, where each layer is enriched by the last. Crucially, beyond dimension one, all $n$-morphisms (for $n > 1$) are invertible. For brevity, we will refer to $(\infty,1)$-categories as $\infty$-categories.}, rather than an $\infty$-groupoid. Examples of such type theories include:

\begin{enumerate}
	\item The $2$-dimensional type theory (2DTT) of Licata and Harper \cite{daniel_r_licata_2-dimensional_2011} addresses asymmetric relations by requiring explicit \textit{variance} annotations on types and terms. This system enforces that covariantly introduced terms are used only covariantly, and similarly for contravariant terms, enabling statements like ``$\Pi$ is covariant in its range and contravariant in its domain."  2DTT diverges from MLTT by omitting the identity type constructor, instead employing a \textit{term transformation} judgment for asymmetric relations. This is coupled with a coercion operation along type families, acting as a unidirectional analog of transport in HoTT. Defining a type in 2DTT involves specifying its formation rule, term introduction/elimination rules ($M:A$), transformation introduction/elimination rules ($\alpha : M \implies_A M'$), and coherence data with the type theory's substitution mechanism.
	\item Building upon 2DTT, the directed type theory of Andreas Nuyts \cite{nuyts2015towards} extends the base type theory to incorporate the full power of HoTT, rather than just MLTT without identity types. A key feature of Nuyts' theory is the explicit articulation of four possible relationships between $f(a)$ and $f(b)$ for a morphism $a \to_A b$ and a function $f : A \to C$: invariance (no relation), isovariance ($f(a) =_C f(b)$), covariance ($f(a) \to_C f(b)$), and contravariance ($f(b) \to_C f(a)$). This results in four variances to track, compared to the two in 2DTT.
	\item In contrast to 2DTT's reliance on a morphism judgment, Page North's type theory \cite{north_towards_2019} introduces a dedicated morphism type. This shift allows morphisms to be treated as first-class citizens within the type theory, enabling the formation of types of morphisms. Furthermore, it provides elimination principles for this morphism type, offering a directed way to reason about morphisms, akin to a directed form of identity elimination. However, similar to 2DTT, the burden of managing variance remains with the user.
  \end{enumerate}

In contrast to the view that all types correspond to $(\infty,1)$-categories, the second perspective posits that only *some* types inherently possess the structure of $\infty$-categories. Consequently, in such type theories, the need for explicit variance annotations and manual variance management is eliminated. Currently, these approaches are primarily derived from Simplicial Type Theory (STT) \cite{riehl_type_2017}, which will be the central focus of this thesis.

STT advances the landscape of directed type theory through the addition of a metatheoretic layer comprising of \textit{shapes} and \textit{extension types}. Shapes are conceptually thought of as the polytopes that are formed within cubes. Extension types can be thought of as a function type where the domain is a shape and which the behavior of maps on a particular subshape is known. This mechanism allows us to ``carve out" specific structures within types, such as morphisms (1-cells) and 2-morphisms (2-cells), reflecting the simplicial nature of the theory. Furthermore, STT enables the formal specification of important type classes. For instance, Segal types, which encode a weak composition principle, can be defined, and their inhabitants naturally exhibit the coherence data characteristic of \textit{categories}. Another crucial example is that of \textit{covariant type families}. These families, given a morphism $f : \ho[A]{a}{b}$ and an element $u : C(a)$, provide a mechanism to lift the morphism $f$ to a corresponding morphism $\trans{f}{u}:\ho[C(b)]{u}{f_*(u)}$, effectively behaving as covariant fibrations.

As it stands, STT provides a strong foundation for addressing the formal category-theoretical questions that arise within higher category theory. This thesis leverages Simplicial Type Theory, assuming the existence of a directed univalence principle, to explore its consequences for the formalization of basic category theory. The primary contribution of this work is the examination of several directed higher inductive types, focusing on their construction and properties. We explore the construction of fundamental examples such as the directed interval and the simplicial circle, aiming to provide insights into their behavior within STT.

This thesis investigates the power and versatility of directed higher inductive types (DHITs) within the framework of simplicial type theory. In Section 1, we warm up with a recapitulation of Homotopy Type Theory (HoTT), providing context for the reader unfamiliar with the subject; however, we will not present any HoTT proofs already established in the literature, instead pointing the reader to \cite{program_homotopy_2013} for a thorough treatment. 

Section 2 introduces the core machinery of simplicial type theory, largely following \cite{riehl_type_2017}. We begin with the foundational concepts of shapes and extension types and proceed to define fundamental type classes, including Segal, Discrete, and Rezk types, which correspond to pre-categories, groupoids, and univalent categories, respectively. We also introduce covariant type families, and, departing slightly from \cite{riehl_type_2017}, we borrow from \cite{buchholtz_synthetic_2022} to define inner type families, which lift 2-morphisms. Crucially, we introduce $\uc$, our Segal directed univalent universe of discrete types, and prove several important lemmas that will be used in the subsequent section. 

Finally, Section 3 constitutes the heart of this thesis, where we discuss DHITs. We demonstrate that DHITs, with their point, morphism, and Segal constructors allow us to specify free categories. Furthermore, we highlight the critical role of inner type families in preserving categorical structure when eliminating out of DHITs. We apply this framework to a detailed study of the directed interval and the simplicial circle, showcasing the practical utility of DHITs. While we focus on these specific examples, the techniques developed herein, particularly the interplay between inner fibrations and DHIT elimination, are general and can be applied to a wider range of free categories, opening exciting avenues for future research in directed type theory.

\end{document}