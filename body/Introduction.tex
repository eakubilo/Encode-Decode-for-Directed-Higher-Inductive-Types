\documentclass[main.tex]{subfiles}
\begin{document}
Homotopy Type Theory (HoTT) offers a new perspective on the foundations of math by blending topological notions with type theory.
HoTT extends Martin-Lof Type Theory (MLTT) with Voevodsky's \textit{univalence axiom} and higher inductive types. Univalence formalizes the common,
but informal, action in math of identifying isomorphic structures as equal. Higher inductive types allow for a concise framework
for defining basic spaces such as the interval, circles, spheres, the torus and complex objects such as homotopy pushouts and cell complexes.
Informally, types in HoTT are seen as topological spaces and proofs of equality are seen as paths in those space. These structures
 make HoTT a powerful language for computer-formalized synthetic homotopy theory.

While HoTT excels at capturing symmetric notions, representing asymmetric structure, such as those that arise in category theory,
is a challenge. \todo{do something hard with asymmetry that would be easy in shott with dhit}

Simplicial Homotopy Type Theory (sHoTT), as developed by Riehl and Shulman [RS17], addresses this by developing
a layered type theory that encompasses both the symmetric notions of HoTT and the asymmetric notions of higher category theory. This thesis explores an 
extension of sHoTT with the additional structure of directed univalence and directed higher 
inductive types, enabling the construction of fundamental categorical objects such as the directed interval and the simplicial circle. Furthermore, 
we investigate inner fibrations, a class of type families that exhibit desirable lifting properties analogous to those in classical category theory. 
We begin by reviewing the essential components of HoTT necessary for understanding sHoTT (Sections 1). Then, we introduce sHoTT and our
 extensions, including a directed univalent universe  $\uc$ and inner fibrations (Section 2). Finally, we develop the theory of directed higher inductive types
  within this enhanced framework (Section 3).
\end{document}