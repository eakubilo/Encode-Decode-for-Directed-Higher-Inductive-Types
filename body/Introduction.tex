\documentclass[main.tex]{subfiles}
\begin{document}
Martin-L\"of Type Theory \cite{martin1984intuitionistic}, also known as Intuitionistic Type Theory but hereafter denoted MLTT, is a constructivist formal logic system that offers an alternative foundation to set theory. In MLTT, the primitive notion of a collection is no longer a \textit{set} and is rather referred to as a \textit{type} satisfying its own rules. With the help of the curry howard isomorphism, the type theory combines mathematical logic and programming languages through its \textit{proofs} as \textit{programs} paradigm. The system also takes \textit{type specifications} to be \textit{propositions}. As a result, the typing judgement 
\[ a : A \]
is also read as
\[
a \text{ is a proof of the proposition } A.
\]
This means the type theory also offers the machinery for doing \textit{proof-relevant mathematics}, by offering a way to speak about ``proof objects'' that is not available in classical mathematics. The analogy of types as propositions does not just end with types. Another basic fixture in the type theory is that of \textit{type-indexed families of types}, \textit{family} or \textit{type family} for short, whose typing judgement comes in the following form:
\[
x : A \vdash B (x) : \U.
\]
Type families are also interpreted as a \textit{hypothetical judgement}, stating that $B(x)$ is a proposition under the assumption $x:A$. 

One other quirk of the type theory has to do with equality. Equality in MLTT comes in two flavors. The first concerns computation. We write \[ a \equiv b : A\] and say that $a$ \textbf{is definitionally equal to} $b$. This is a sort of equality that happens \textit{a priori}, by definition and thus requires no proof. With definitional equality, if we know that $a \equiv b$, then we can automaticaly lift that to an equality between types $B(a) \equiv B(b)$. 

The second concerns \textit{provability}. There are times when we want to treat two objects with the same properties as equal. For example, in category theory equality does not only happen by definition. It is common to treat \textit{equivalent} objects as being equal. Those, most categorical definitions are up to ``canonical isomorphism'', or \textit{equivalences}. For example, the categories $A \times B$ and $B \times A$ are not literally the same, but it is common to treat them as if they were. In our type theory, we consider such an operation of considering two things the same as ``identification'',  and write the collection of proofs of identity of two terms as \[x=_A y : \U\]
and say that $x$ \textbf{is propositonally equal to} $y$. Our identity types truly capture the notion of equality, as we can can define terms showing the identity type is an equivalence relation:
 $$\refl{\text-} : \prd{x:A}a=_A a,$$ $$(\text-)^{-1}:\prd{x,y:A}(x =_A y ) \to (y =_A x),$$ and  $$\text-\circ\text- : \prd{x,y,z : A} (x =_A y) \to (y =_A z) \to (x =_A z).$$
These terms form data witnessing that identity types are reflexive, symmetric, and transitive respectively.

Leibniz law, or the indescernability of identicals, is also another primitive feature of our type theory. In short, proofs of equality $p : x =_A y$ induce a coercion of type families $B(x) \to B(y)$. In fact, the coercion is actually an equivalence $B(x) \simeq B(y)$. 

The type theory also has more direct connections with category theory. Observed first by Hofmann and Streicher \cite{hofmann1998groupoid}, MLTT types have the structure of groupoids, which are categoreies where all of the morphisms are isomorphisms. In the interpretation, the type $x=_A y$ becomes the hom-set $\ho[A]{x}{y}$\footnote{This is meant categorically, not type theoretically.}. Identity types satisfy more coherence rules, such as unit laws and associativity, which serve to make our interpretation of the types as groupoids in a category well-founded. In such an interpretation, though, we note some tension as the coherence laws do not hold up to strict equality. As mentioned in \cite{berg_garner_richard}, associativity is a prime example of this as the diagram:
\[\begin{tikzcd}
	{(x=_A y) \times( y =_A z) \times (z =_A w)} && {(x=_A y) \times (y=_Aw)} \\
	\\
	{(x =_A z )\times (z =_A w)} && {x=_A w}
	\arrow[from=1-1, to=1-3]
	\arrow[from=1-1, to=3-1]
	\arrow[from=1-3, to=3-3]
	\arrow[from=3-1, to=3-3]
\end{tikzcd}\]
does not commute on the nose, but rather up to equivalence. Thus, types can have much more rich structure than the notion of a groupoid captures. As \textit{op. cit.} shows, types are $\omega$-groupoids. An $\omega$-groupoid can informally be thought of as a category where the morphisms have a dimension property and higher morphisms are built out of lower ones, all morphisms invertible. More specifically, we can borrow language from Topology and consider objects to be $0$-cells. Then, morphisms between $0$-cells are $1$-cells, morphisms between $1$-cells are $2$-cells, ad infimum. With such structure, two important questions are borne: 
\begin{enumerate}
  \item When does an equivalence of types $B(x) \simeq B(y)$ induce an equality of types $B(x)=_{\U} B(y)$?
  \item How are the basic cell-complexes of topology, like the circle or torus defined?
\end{enumerate}

While MLTT isn't expressive enough to generally answer these questions, we can extend the type theory with Voevodsky's univalence axiom and higher inductive types, which positively addresses the matter. Such an extension of MLTT with these two fixtures is called Homotopy Type Theory, or HoTT for short. Voevodsky's univalence axiom formalizes the notion of ``equivalent structures are equal''. As a consequence of the axiom, we can assert that type families \textit{do} act functorially with respect to propositonal equality. Using univalence, we can conclude that the equivalence $B(x) \simeq B(y)$ is in fact an equality $B(x) = B(y)$. Higher inductive types extend the inductive type schema to allow for \textit{path} constructors, not just point constructors. This allows for a logical framework for defining common spaces such as the circle and torus as mentioned before, and even more complex constructions like homotopy pushouts and colimits. 

The additional structures to MLTT to create HoTT are intended to force the interpretation of types roughly as the homotopy types of spaces. Thus, the type theory lends itself well to a homotopical interpretation. An important notion in homotopy theory is that of a \textit{fibration}. A fibration $p : E \to B$ is a functor between spaces that has certain lifting properties. A common lifting property classifies fibrations that lift a path in $B$ to a path in $E$. Our attitude in HoTT allows us to then consider type families $P : A \to \U$ not just as functors, but as \textit{fibrations}. We do that in type theory by noting that type families induce maps from the total space, $\sm{x:A}P(x)$, to the base space, $A$. Thus, an important property of type families is that they lift paths in $A$ to paths in $\sm{x:A}P(x)$.


While HoTT is powerful, it has its own pitfalls, especially when it comes to modeling asymmetric relations. To illuminate this pitfall, we follow \cite{2DTT_Licata_Harper, gratzer_directed_2024} and study a formalization of the $\Li$ inductive family. We start by defining the type of monoids, which are types equipped with a binary operation satisfying associativity and identity axioms. Formally, we write:
\[
\mon \defeq \sm{A : \U}{\star : A \times A \to A}\assoc(A, \star) \times \iden(A, \star)
\]
with $\assoc$ defined as
\[
\assoc \defeq \prd{A : \U}{\star : A \times A \to A}{a,b,c : A} a \star (b \star c) = (a \star b) \star c
\]
and $\iden$ defined as
\[
\iden \defeq \prd{A : \U}{\star: A \times A \to A}\sm{e : A}\prd{a : A} (e \star a = a) \times ( a \star e = a).
\]
As the names imply, $\assoc$ and $\iden$ express that its input either satisfies associativity or has an identity. With these in hand, we move to defining for every monoid $A : \mon$, an inductive type $\Li\,A$ generated by:
\begin{itemize}
  \item An element $\nil : \Li\,A$
  \item For every element $l : \Li\,A$ and $a : \fst(A)$ an element $\cons{a}{l} : \Li{A}$.
\end{itemize}
We then finish our setting the scene by defining a function $$\add : \prd{A : \mon}\Li{A} \to A$$ by $\Li$ recursion. With our scene set, we can start to ask important questions like ``what structures respect $\add$''? With little work, we can show that equivalence relations respect $\add$. For example, it is a basic exercise to take an equality between monoids $A = B$ and reflect that as an identity of sums $\add\,A = \add\,B$. Our classical mathematical intuition allows us to conjecture that the claim can be generalized to instead state that \textit{preorders}, which are \textit{reflexive} and \textit{transitive} relations, respect $\add$. 

To try to prove our conjecture in HoTT, we first assume that the types $\mon$ and $\Li{A}$ are \textit{categories}, as defined in \cite[9]{program_homotopy_2013}. Given a monoid morphism $f : \ho[\mon]{A}{B}$, it can be observed the type $\ho{\Li{A}}{\Li{B}}$ is not automatically inhabited. We first would have to prove that $\Li$ is a \textit{functor}, as \textit{op. cit.} defines. Moreover, noting the parametricity of $\add$, we might also hope that we can obtain a proof of our conjecture as a ``theorem for free'' \cite{wadler1989theorems}. To state what that theorem might be, we would first need to define a map
$$f_* : \Li{A} \to \Li{B}.$$
Once we've defined such a map, the ``free theorem'' should be an inhabitant of the type:
$$\add{B} \circ f_* = f \circ \add{A}.$$
Although all of this is theoretically possible, it is not as simple of a job as showing that $\add$ reflects equality. There is a large contrast between the ease of proving equivalence relation-like statements and the effort required to show preorder-like statements. The lack of power for defining asymmetric relations is well-highlighted by the lack of an ability to relate functions between types to morphisms between types, and the lack of ability to define categories through a similar logical framework as HITs.

To reduce some of this tension, \textit{directed type theory} has been explored. It comes in two main strains. The first, which does not include Simplicial Type Theory, is one where the vantage point is that types correspond to $(\infty,1)$-\textit{categories}\footnote{An $(\infty,1)$-category is a category with an infinite hiearchy of morphisms, each layer enriched by the last. It also has the additional property that $n$-morphisms, for $n>1$, are all invertible. We will refer to $(\infty,1)$-categories as $\infty$-categories.}, rather than $\infty-$groupoids. This includes:
\begin{enumerate}
  \item The $2$-dimensional type theory of Licata and Harper \cite{2DTT_Licata_Harper}. The type theory 
\end{enumerate}

\end{document}