\documentclass[main.tex]{subfiles}

\begin{document}
A type theory with shapes is a layered type theory consisting of a \textbf{cube layer}, a simple intuitionistic type theory with finite products,
a \textbf{tope layer}, a coherent first order logic over the cube layer, and a \textbf{type layer}, a dependent type theory that contains additional
contexts for the cube and tope layer. sHoTT then builds on this foundation by axiomatizing a strict interval cube $\mathbbm{2}$, an inequality tope,
and rules ensuring the tope layer with the inequality tope forms the coherent theory of the strict interval.

Specifically, the \textbf{cube layer} is specified by the following formal rules
\begin{mathpar}
    \inferrule{ }{1 \,\cu} \and
    \inferrule{ I \, \cu \\ J\, \cu}{I \times J \, \cu }\and
    \inferrule{(t:I) \in \Xi}{\Xi \vdash t : I}\and
    \inferrule{ }{\chi \vdash \star : 1} \and
    \inferrule{\Xi \vdash s : I \\ \Xi \vdash t : J }{\Xi \vdash \langle s,t \rangle : I \times J}\and
    \inferrule{\Xi \vdash t : I \times J}{\Xi \vdash \pi_1(t) : I}\and
    \inferrule{\Xi \vdash t : I \times J}{\Xi \vdash \pi_2(t) : J}
\end{mathpar}
The \textbf{tope layer} builds on top of this. The intuition for the tope layer comes from viewing it as "polytopes" embedded in cubes.
The formal rules for the tope layer are as follows

\end{document}