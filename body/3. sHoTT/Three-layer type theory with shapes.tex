\documentclass[main.tex]{subfiles}

\begin{document}
The foundation of sHoTT is a \textit{three-layer type theory with shapes}. More specifically, it is a type theory that consists 
of a \textbf{cube layer} at the top, which is a simple type theory with finite products. Cubes in sHoTT are finite powers of 
$\mathbbm{2}$, such as those in figure\todo{draw important cubes}. 

We have the \textbf{shape layer} sitting under the cube layer. The shape layer allows us to talk about polytopes that live in cubes.
A polytope can be thought of as a shape composed out of simplices. That is, it is a shape composed of lines, triangles, tetrahedra
and so on. Examples of shapes can be seen in \todo{draw important shapes that we end up using}
\end{document}