\documentclass[main.tex]{subfiles}

\begin{document}
With our three-layer type theory and extension types, we can start to define familiar category theoretic notions. We start by defining important types which depend on shapes.

\begin{definition}
    For a type $A$ and points $x,y : A$ which induce a term $[x,y]$\footnote{Spelled out, this is a term that depends on $t : \pardo $ that evaluates to $x$ when $t \equiv 0$ and $y$ when $t \equiv 1$.} in context $\pardo$, we write
    $$\ho[A]{x}{y} \defeq \ndexten{\osx}{A}{\pardo}{[x,y]}$$
    and say that an element of $\ho[A]{x}{y}$ is a \textbf{morphism} from $x$ to $y$. 
\end{definition}
\begin{definition}
    For points $x,y,z : A$, and morphisms $f : \ho[A]{x}{y}$, $g : \ho[A]{y}{z}$, and $h : \ho[A]{x}{z}$ which induces a term $[x,y,z,f,g,h]$\footnote{Similarly, this is a term that depends on $\pair{t}{s} :\pardt$ that evaluates to:
    \begin{itemize}
        \item $x$ when $\pair{t}{s} \equiv \pair{0}{0}$ 
        \item $y$ when $\pair{t}{s} \equiv \pair{1}{0}$ 
        \item $z$ when $\pair{t}{s} \equiv \pair{1}{1}$
        \item $f$ when $\pair{t}{s} \equiv \pair{t}{0}$
        \item $g$ when $\pair{t}{s} \equiv \pair{1}{s}$
        \item $h$ when $\pair{t}{s} \equiv \pair{t}{t}$
    \end{itemize} } in context $\pardt$, we write
    $$\homtwo{A}(x,y,z,f,f,h) \defeq \ndexten{\tsx}{A}{\pardt}{[x,y,z,f,g,h]}$$
    or, in an abbreviated fashion,
    $$\homtwoshort{A}(x,y,z,f,g,h)$$
    and say that an element of $\homtwoshort{A}(x,y,z,f,g,h)$ is a $2$-\textbf{morphism}.
\end{definition}

\begin{remark}
    We will mostly write a 2-morphism as 
\end{remark} 

Not only do functions act functorially on paths, they also act functorially on morphisms and higher morphisms.
\begin{lemma}(Morphism application)
    Let $A,B$ be types and $f : A \to B$. Then, given $x,y : A$, there are functions 
    $$\maparrfunc{f} : \ho[A]{x}{y} \to \ho[A]{f x}{f y}$$
    %ap^2
\end{lemma}

In section 2, we saw that composition of paths was a function. With morphisms, we don't get the same luxury a priori. Instead, we can ask for types whose morphisms do have composites. So, composition of paths is a parametrically polymorphic function while composition of morphisms is a type class polymorphic function. That leads us to the following definition.
\begin{definition}\label{def:def3.3}
    Let $A$ be a type, $x,y,z : A$ and $f : \ho[A]{x}{y}$ and $g : \ho[A]{y}{z}$. We say $A$ is \textbf{Segal} if the type $${\sm{h : \ho[A]{x}{z}} \hot[A]{f}{g}{h}} $$
    is contractible. The unique inhabitant is denoted by $(g\circ f, \comp{g}{f}).$
\end{definition}

In other words two composable morphisms are enough to specify a 2-morphism in a Segal type. That leads us to the following important alternative characterization of Segal-ness.

\begin{lemma}
A type $A$ is Segal if and only if the restriction map
$$
(\tsx \to A) \to (\tohorn \to A) 
$$
is an equivalence. 
\end{lemma}
\begin{remark}
    In the proof of (lemma reference), c.f proves along the way that there is an equivalence
    \begin{equation}
        \sm{h : \ho[A]{x}{z}}\homtwo{A}(x,y,z,f,g,h) \simeq \ndexten{\tsx}{A}{\tohorn}{[x,y,z,f,g]}.
    \end{equation}

    \noindent The same argument applies to form equivalences 

    \begin{equation}
        \label{rem:sumoverfistt}
    \sm{f : \ho[A]{x}{y}}\homtwo{A}(x,y,z,f,g,h) \simeq \ndexten{\tsx}{A}{\tthorn}{[x,y,z,g,h]}
    \end{equation}
    and
    \begin{equation}
    \sm{g : \ho[A]{y}{z}}\homtwo{A}(x,y,z,f,g,h) \simeq \ndexten{\tsx}{A}{\tzhorn}{[x,y,z,f,h]}.
    \end{equation}
\end{remark}
Paths between morphisms also correspond two 2-morphisms:
\begin{lemma}
    \label{lem:pathis2mor}
    For any $f,g : \ho[A]{x}{y}$ in a Segal type $A$, the natural map
    $$f = g \to \homtwo{A}(x,y,z, \idhom{x}, f,g)$$
    is an equivalence.
\end{lemma}
\begin{lemma}
    \label{lem:compequalis2mor}
    For any $f: \ho[A]{x}{y}$, $g : \ho[A]{y}{z}$, and $h : \ho[A]{x}{z}$ in a Segal type $A$, the natural map
    $$g \cdot f = h \to \homtwo{A}(x,y,z,f,g,h)$$
    is an equivalence.
\end{lemma}
\end{document}