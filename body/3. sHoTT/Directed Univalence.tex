\documentclass[main.tex]{subfiles}

\begin{document}
For discrete types A and B, we seek to make an identification between morphisms $\ho[\uc]{A}{B}$ and functions $A \to B$. Just as we do so in informal HoTT, we can make such an identification by defining some canonical function:

\begin{lemma}
\label{lem:can_define_htf}
For any two discrete types $A$ and $B$, there is a function
\begin{equation}
    \htf : \ho[\uc]{A}{B} \to (A \to B).
\end{equation}

\end{lemma}
\begin{proof}
    We first observe that the type family \begin{equation}
    \label{eqn:zetaisucovid}
        \zeta \defeq \lam{B : \uc} x : \uc \to \uc
    \end{equation}
    is covariant, so by Lemma~\ref{lem:covcodiscov} the type family $$C \defeq \lam{B : \uc}A \to \zeta(B) : \uc \to \U$$
    is also covariant. Note that \eqref{eqn:zetaisucovid} means that $C \equiv \lam{B : \uc} A \to B$. Finally, we appeal to the Yoneda lemma to define $\htf$:
    $$\htf \defeq  \yon{C}{A}(\id{A})(B) \defeq \lam{f}f_*(\id{A}) : \ho[\uc]{A}{B} \to (A \to B) $$
\end{proof}
\begin{remark}
    Note that $\htf$ acts functorially, as for any $f : \ho[\uc]{A}{B}$ and $g : \ho[\uc]{B}{C}$,
    \begin{align*}
        \htf(g \cdot f) \equiv\\
        (g\cdot f)_*(\id{A}) =\\
        g_*(f_*(\id{A}))
    \end{align*}
with the second equality following from Proposition 8.16. 

\end{remark}
Just as HoTT does not come equipped with tools to show idtoeqv is an equivalence, sHoTT also does not come equipped with the tools to show {\htf} is an equivalence. Instead, we can postulate \textit{directed univalence} as an axiom.

\begin{axiom}[Directed Univalence]
    For any discrete types $A,B : \uc$, the function defined in Lemma~\ref{lem:can_define_htf} is an equivalence.
\end{axiom}

For discrete types $A$ and $B$, we can break the equivalence up into:
\begin{itemize}
    \item An introduction rule for $\ho[\uc]{A}{B}$ denoted $\dua$ for "directed univalence axiom":
    $$\dua : A \to B \to (\ho[\uc]{A}{B})$$
    \item An elimination rule
    $$\htf \defeq \yon{C}{A}(\id{A}) : \ho[\uc]{A}{B} \to (A \to B)$$
    \item The propositional computation rule
        $$\htf(\dua(f)) = f$$
    \item The propositional uniqueness rule, which states for any $f : \ho[\uc]{A} {B}$,
    $$f = \dua(\htf(f))$$
\end{itemize}
Directed univalence also allows us to  identify identity morphisms with identity functions and composition of morphisms in $\uc$ with composition of functions, stated tersely as
\begin{align*}
    \idhom{A} = \dua(\id{A})\\
    \dua{g}\cdot\dua{f} = \dua(g \circ f).
\end{align*}
The first equality from the fact that $\id{A} = (\idhom{A})_*(\id{A}) \equiv \htf(\idhom{A})$ by proposition 8.16. To show the second, if we define $p \defeq \dua(f)$ and $q \defeq \dua(g)$, then
    $$\dua(g \circ f) = \dua(\htf(p) \circ \htf(q)) = \dua(\htf(p \cdot q)) = p \cdot q$$

Finally, the directed univalence even extends to characterizing dependent functions as morphisms in $\uc$.
\begin{lemma}
    For any type family $P : A \to \uc$, for all $x : A$ and $Q : \tilde{P} \to \uc$, there is an equivalence

    $$\prd{p : P(x)} Q(x,p) \simeq \sm{f : \ho[\uc]{P(x)}{\sm{p : P(x)} Q(x,p)}} \dua(\fst) \circ f = \idhom{P(x)}, $$
    where $\fst : \sm{p : P(x)}Q(x,p) \to P(x)$
\end{lemma}
\begin{proof}
    \label{lem:covdomcovcodisinner}
    We begin by noting the following equivalences
    \begin{align*}
        \sm{f : \ho[\uc]{P(x)}{\sm{p : P(x)} Q(x,p)}} \dua(\fst) \circ f = \idhom{P(x)} \simeq \\
        \sm{f : P(x)\to \sm{p : P(x)} Q(x,p)}\dua(\fst) \circ \dua(f) = \idhom{P(x)}\simeq \\
        \sm{f : P(x)\to \sm{p : P(x)} Q(x,p)}\dua(\fst \circ f) = \idhom{P(x)}\simeq \\
        \sm{f : P(x)\to \sm{p : P(x)} Q(x,p)} \fst \circ f = \id{P(x)}
    \end{align*}
    with the first equivalence following from exercise 2.10, the second from the functorality of $\dua$ and the third coming from theorem 4.6.3. Now that we are working with functions, we can reduce the types further:
    \begin{align*}
       \sm{f : P(x) \to \sm{p : P(x)}Q(x,p)}\fst \circ f = \id{P(x)} \simeq \\
       \sm{(f_1,f_2) : \sm{g : P(x) \to P(x)}\prd{p : P(x)}Q(x,g(p))}f_1 = \id{P(x)} \simeq\\
       \sm{f_1 : P(x) \to P(x)}\sm{f_2 :\prd{p : P(x)}Q(x,f_1(p))} f_1 = \id{P(x)} \simeq \\
       \prd{p : P(x)}Q(x,p)
    \end{align*}
    with the first equivalence following from the type theoretic axiom of choice, the second following from the associativity of sigma types and the third following from singleton contractibility. Thus, we can get our desired equivalence by composing the two equivalences formed in this proof.
    \end{proof}



\end{document}