\documentclass[main.tex]{subfiles}
\begin{document}
\subsection{Introduction}

\subsection{Unit}
We begin by defining the type Unit, which only has one constructor:
\begin{itemize}
    \item $\star\,: \ $Unit
\end{itemize}
Abstractly, this should be the type generated by a single point. By definition, we can always form the morphism $\id{\star} : \ho[\text{Unit}]{\star}{\star}$. We expect, though, for there to be no more information than we put into the type. So, the identity should be the only morphism of the unit type.
\begin{lemma} For all $ \normalfont f : \ho[\text{Unit}]{\star}{\star}$,
\[ f = \id{\star}\]
\end{lemma}
\begin{proof}
For all $x : \mathbbm{2}$, we have that $f(x) = \star = \id{\star}(x)$. By functional extensionality, $f = \id{\star}$.
\end{proof}
The unit type corresponds to the category with one object and one morphism (the identity).
\subsection{I}
The free category with a arrow $I$, is generated by the following constructors.
\begin{itemize}
    \item $0_I$
    \item $1_I$
    \item $\seg : \ho[I]{0_I}{1_I}$
    \item $\text{is-segal}(I)$
\end{itemize}
We then check that the positive description of the type is enough to ensure that the type actually models the free category with an arrow. To start, we check that the only morphisms starting from $0_I$ are the identity morphism and $\seg$.
\begin{lemma}
\[
\prod_{x:I}\ho[I]{0_I}{x}\text{ \emph{is contractible}}
\]
\end{lemma}
\begin{proof}
    By Lemma 3.11.6 in HoTT book, it suffices to show that each fiber of $\lambda x. \ho[I]{0_I}{x}$ is contractible. So, we first define a map $$\text{center} : \displaystyle \prod_{x:I}\ho[I]{0_I}{x}.$$
    which gives the center of contraction for each fiber as follows:
    \begin{align*}
        &center(0_I) = \id{0_I}\\
        &center(1_I) = \seg\\
        &center(\seg) = \land_\seg
    \end{align*}
    It then suffices to inhabit the type
    $$\prod_{x:I}\prod_{q:\ho[I]{0_I}{x}}q=\text{center}(x).$$
    Because of the equivalence from theorem 9.5 of Riehl Shulman, (the dependent yoneda lemma), to exhibit such a type we can give an inhabitant of $\id{0_I}=\text{center}(0_I) \equiv \id{0_I}$, of which $\refl{\id{0_I}}$ suffices.
\end{proof}
We can similarly characterize the morphisms starting from $1_I$. Assuming one the existence of one particular type family allows us to characterize such morphisms.
% \begin{lemma}
%     The type family $$\emph{code} : I \to \mathcal{U_{\emph{cov}}}$$
%     \begin{align*}
%         &\emph{code}(0_I) = \emph{void}\\
%         &\emph{code}(1_I) = \emph{Unit}\\
%     \end{align*}
%     is covariant.
% \end{lemma}
% \begin{proof}
%     To show that $\text{code}$ is a covariant type family, we first start by showing it has
% \end{proof}
\begin{lemma}
    Given a covariant type family $$\emph{code} : I \to \mathcal{U_{\emph{cov}}}$$
    defined as
    \begin{align*}
        &\emph{code}(0_I) = \emph{void}\\
        &\emph{code}(1_I) = \emph{Unit}\\
        &\emph{code}(\emph{\seg}) = \emph{duahom(!)}
    \end{align*}
    then
    $$\prod_{x:I}\ho[I]{1_I}{x}\cong \emph{code}(x)$$
\end{lemma}
\begin{proof}
    By lemma 9.10 of riehl shulman, it suffices to show that the category of elements has an initial object. We can show that $(1_I, \star)$ is an initial object of the category $$\sum_{x:I}\text{code}(x).$$ That is, we can define an element of the type family
    $$(1_I,\star)\text{-is-initial}:\prod_{(a,u):\sum_{x:I}\text{code}(x)}\ho[\sum_{x:I}\text{code}(x)]{(1_I,\star)}{(a,u)} \text{ is contractible}$$
    By $\prod$-induction, we first consider $(0_I, u)$. Since $u : \text{void}$, we can simply return abort(u). For the pair $(1_I, \star)$, note that $\ho[\text{Unit}]{\star}{\star}$ is contractible by lemma 1. So, it suffices to show that $\ho[I]{1_I}{1_I}$ is contractible. For any $f : \ho[I]{1_I}{1_I}$, we can form the degenerate $2$-simplex $\lambda\langle t, s \rangle.f(t) : $hom$^2_I$(
    \begin{tikzcd}[every label/.append style = {font = \tiny} ,sep=small ]
	{\scriptstyle 1_I} & {\scriptstyle 1_I} \\
	{\scriptstyle 1_I} & {}
	\arrow["{\text{id}_{1_I}}", from=1-1, to=1-2]
	\arrow["f"', from=1-1, to=2-1]
	\arrow["{\text{id}_{1_I}}", from=1-2, to=2-1]
\end{tikzcd}). Since $I$ is segal, this is equivalent to a map $p : f = \id{1_I}$ by lemma 5.10 of Riehl Shulman. Thus, $\ho[I]{1_I}{1_I}$ is contractible.
\end{proof}
\end{document}