\documentclass[main.tex]{subfiles}

\begin{document}
\todo{do codes for sigma and pi}
In section 2, we saw that all type families were fibrations and what that means specifically in HoTT. Of course in sHoTT type 
families are also fibrations, with respect to \textit{paths} that is. When discussing which type families lift simplicial shapes,
we must specify which shapes are being lifted. An important collection of type families are the ones that lift \textit{morphisms},
covariantly or contravariantly. Before we can precisely define such type families, we must first specify what it means to be a 
 ``lifted morphism".
\begin{definition}[\cite{riehl_type_2017}, 8.1]
Given a type $A$, a type family $C : A \to \U$, points $x,y : A$, a morphism $f : \ho[A]{x}{y}$ and points $u : C(x)$ and $v : C(y)$, the type of \textbf{dependent morphisms over} $f$ \textbf{from} $u$ to $v$ is written as $$
\dho{C}{f}{u}{v} \defeq \exten{t : \osx}{C(f(t))}{\pardo}{[u,v]}.$$
\end{definition}
With the dependent morphism defined, we can show that dependent functions can also be definitionally applied to morphisms.
\begin{lemma}
    Let $B : A \to \U$ and $\phi : \prd{x : A}B(x)$. Then, given $x,y : A$, there is a function
    $$\mapdepfunc{\phi} : \prd{f : \ho[A]{x}{y}}\ho[B(f)]{\phi(x)}{\phi(y)}$$
\end{lemma}
\begin{proof}
Same as \cref{lem:mapfunc}.
\end{proof}

The dependent morphism represents the intuitive notion of ``morphism in the total space". It is instructive to view an inhabitant $g :\dho{C}{f}{u}{v}$ as literally living above the morphism $f$, as the following diagram illustrates:

\begin{figure}[H]
    \includesvg{figures/covariant.svg}
\end{figure}

We do not know without extra information that $C$ depends functorially on $A$. So, we can instead restrict ourselves to the study of type families that satisfy do act as covariant functors.
\begin{definition}[\cite{riehl_type_2017}, 8.2]
For any type $A$, a type family $C : A \to \U$ is \textbf{covariant} if for every $f : \ho[A]{x}{y}$  and $u : C(x)$, the type $$\sm{v : C(y)} \dho{C}{f}{u}{v}$$
is contractible. When the context is clear, we write the specific inhabitant as $(\dtran{C}(f, u), \trans{f}{u})$ or 
even more tersely as $(f_*u, \trans{f}{u}).$

Dually\footnote{We will not explicitly state dual theorems and definitions when they are obvious from context.}, $C$ is \textbf{contravariant} if for every $f : \ho[A]{x}{y}$  and $v : C(y)$, the type $$\sm{v : C(x)} \dho{C}{f}{u}{v}$$
is contractible.
\end{definition}
We can also alternatively, but perhaps more traditionally, characterize covariant fibrations as type families that satisfy a lifting property. A covariant fibration not only lifts paths, but lifts them uniquely. The following diagram encapsulates this:

\begin{figure}[h]
    \includesvg{figures/uniquelift.svg}
\end{figure}

The dotted line represents the uniqueness of $\trans{f}{u}$ as a morphism starting at $u$ living over $f$. The diagram is thus formalized by the following lemma.
\begin{lemma}[\cite{riehl_type_2017}, 8.4]
    A type family $C : A \to \U$ is covariant if and only if for all morphisms $f : \ho[A]{x}{y}$ and points $u : C(x)$ the type $\exten{t : \osx}{C(f(t))}{0}{u}$ is contractible; there is a unique lifting of $f$ that starts at $u$.
\end{lemma}
\begin{remark}
    Covariant families are stable under substitution. That is, for any $g : B \to A$ and covariant type family $C : A \to \U$, the type family
    $\lam{b}C(g(b))$ is also covariant.
\end{remark}
Having explored various characterizations of the covariant property, we now focus on the properties specific to our covariant type families. To begin, in HoTT, transporting along a type family is equivalent to applying the type family to a path and then coercing along the resulting path. The directed version of the result holds in sHoTT.
\begin{lemma}
    \label{lem:tpthomiscoethenapp}
    Let $C : B \to \U$ be covariant and $h : A \to B$. For any $f : \ho[A]{x}{y}$ and $u : C(h(x))$ we have that
    \[
         \dtran{C\circ h}(f, u) = \dtran{C}(\mapfunc{h}{f}, u)
    \]
\end{lemma}
\begin{proof}
    Note that 
    \[
        (\dtran{C\circ h}(f, u), \trans{f}{u}) : \sm{v : C(h(y))}\ho[(C \circ h)(f)]{u}{v} \equiv 
    \]
    and 
    \begin{align*}
        (\dtran{C}(\mapfunc{h}{f}, u), \trans{\mapfunc{h}{f}}{u}) : &\sm{v : C(h(y))}\ho[C(\mapfunc{h}{f})]{u}{v} \equiv \\
        & \sm{v : C(h(y))}\ho[C(h(f))]{u}{v}
    \end{align*}
    Since $C \circ h$ is covariant, the type $\sm{v : C(h(y))}\ho[C(h(f))]{u}{v}$ is contractible, 
    hence $\dtran{C\circ h}(f, u) = \dtran{C}(\mapfunc{h}{f}, u)$, as desired.
\end{proof}

We can also show that dependent morphisms in a covariant type family from some are equivalent to a path whose domain is a transport of the source of the dependent morphism, analogous to the situation in HoTT.

\begin{lemma}[\cite{riehl_type_2017}, 8.15]
    \label{lem:depmorispath}
    If $C : A \to \U$ is covariant, $f : \ho[A]{x}{y}$, $u: C(x)$, and $v : C(y)$, then
    $$\ho[C(f)]{u}{v} \simeq (f_*u =_{C(y)}v).$$
\end{lemma}

We now shift our focus to the categorical nature of fibrations, particularly in the context of covariant families over Segal bases. Such covariant families exhibit an important property: directed transporting is functorial. This property is formalized in the following lemma.

\begin{lemma}[\cite{riehl_type_2017}, 8.16]
    \label{lem:tptisfunctorial}
    For any Segal type $A$, covariant family $C : A \to \U$, $f : \ho[A]{x}{y}$, $g : \ho[A]{y}{z}$, and $u : C(X)$, we have
    \begin{gather*}
        g_*(f_*\,u) = (g\cdot f)_*\,u \qquad \normalfont \text{ and } \qquad (\idhom{x})_*\,u = u
    \end{gather*}
\end{lemma}

Furthermore, covariant families over Segal bases faithfully represent compositions in the base. This follows from the functoriality of the type family with respect to the base. As a result, the total space of the type family inherits the Segal property, where weak composition in the total space arises naturally from the composition in the base. This is captured in the following lemma:

\begin{lemma}[\cite{riehl_type_2017}, 8.8]
    If $A$ is Segal and $C : A \to \U$ is covariant, then 
    $$\sm{a : A}C(a)$$
    is Segal.
\end{lemma}

Notably, certain type families are automatically covariant when defined over a Segal base. A significant example, drawn from category theory, is the representable type family \(\ho[A]{a}{-} \). In sHoTT, the representable plays an equally critical role. 

One particularly useful property of the representable in sHoTT is its ability to characterize Segal types. This relationship is formalized in the following lemma:

\begin{lemma}[\cite{riehl_type_2017}, 8.13]
    Given a type $A$ and a point $a : A$, then the type family
    $$\lam{x} \ho[A]{a}{x} : A \to \U$$ is covariant if and only if $A$ is Segal.
\end{lemma}

Covariant type families act functorially in other important ways. For example, naturality comes for free, for maps between covariant
 types.

\begin{lemma}[\cite{riehl_type_2017}, 8.17]
    For any two covariant families $C,D : A \to \U$ and a fiberwise map $\phi : \prd{x : A}C(x) \to D(x)$, $f: \ho[A]{x}{y}$ and $u:C(x)$,
    $$\phi_y(f_*\,u) = f_*(\phi_x(u)).$$
\end{lemma}

An important structural property of covariant families over Segal bases is that their fibers are discrete. This result provides a foundational link between Segal bases and the nature of the types within covariant families:


\begin{lemma}[\cite{riehl_type_2017}, 8.18]
    For any Segal type $A$ and covariant map $C : A \to \U$, and $x : A$, the type $C(x)$ is discrete.
\end{lemma}

\begin{corollary}[\cite{riehl_type_2017}, 8.19]
    If $A$ is Segal, then for any $x,y: A$ the type $\ho[A]{x}{y}$ is discrete.
\end{corollary}

\begin{corollary}[\cite{riehl_type_2017}, 8.20]
    If $A$ is discrete, then for any $x,y : A$, so is $x =_A y$.
\end{corollary}

When dealing with multiple argument functions, covariance can be defined:
\begin{definition}
    \label{lem:covdomcovcodiscov}
    Given any type family $P : A \to \U$, a type family $Q : \prd{x : A}P(x) \to \U$ is covariant if the type family
    \[
    C \defeq \lam{(x,p)}Q(x,p) : (\sm{x:A}P(x)) \to \U
    \]
    is covariant.
\end{definition}

We can show that the path types in covariant families are covariant:
\begin{lemma}[\cite{riehl_type_2017}, 8.26]
    \label{lem:pathincoviscov}
    Suppose $C : A \to \U$ is covariant. Then 
    \[
    \lam{a}{u}{u}u=v : \prd{a:A}(C(a) \to C(a) \to \U)
    \]
    is covariant.
\end{lemma} 
The previous lemma is slightly easier to use in the following incarnation:
\begin{corollary}
    \label{cor:pathwithfunciscov}
    Suppose $C : A \to \U$ is covariant and we have maps $u : \prd{x:A}C(v)$ and $v : \prd{x:A}C(x)$ Then 
    \[
    \lam{x}u(x)=v(x) :(A \to  \U)
    \]
    is covariant.
\end{corollary} 
\begin{proof}
    Since $C$ is covariant, 
    \[\lam{a}{u}{u}u=v : \prd{a:A}(C(a) \to C(a) \to \U)\] is covariant by \cref{lem:pathincoviscov}.
    So, by \cref{lem:covdomcovcodiscov}, 
    \[
    \lam{(x,u,v)}u=v : (\sm{x:A}C(x)\times C(x))\to \U
    \] is covariant.
    We can then precompose with the map $x \mapsto (x,u(x),v(x))$ to form the family 
    \[
        \lam{x}u(x)=v(x),
    \]
    with covariance of this family coming from our remark above.
\end{proof}
Mapping into a covariant family is covariant: 
\begin{lemma}[\cite{riehl_type_2017}, 8.30]
    \label{lem:covcodiscov}
    For types $C$, $A$, and covariant family $B : C \to \U$, the type family $$\lam{x : C} A \to B(x) : C \to \U$$
    is also covariant.
\end{lemma}
Mapping out of a covariant family \textit{into a discrete type} is contravariant:
\begin{lemma}[\cite{riehl_type_2017}, 8.31]
    \label{lem:covdomiscon}
    For a discrete type $Y$ and a covariant type family $C : A \to \U$, the type
    $$\lam{x : A}C(a) \to Y$$
    is contravariant.
\end{lemma}


The closure properties also allow us to characterize the action of transport on objects. For example transporting in a pair is, as expected, a pair of transports.
\begin{corollary}
    \label{cor:tptinsigma}
     Given type families $P : A \to \U$, $Q : \prd{x : A}P(x) \to \U$ that are covariant, a morphism $f : \ho[A]{x}{y}$, point $p : P(x)$, and map 
     $\phi : \prd{p : P(x)}Q(x,p)$,
    $$(f_*(p),(f_* \circ \phi)(p)) = f_* (p, \phi(p))$$
\end{corollary}
\begin{proof}
    By \cref{lem:covdomcovcodiscov}, $C \defeq \lam{x}\sm{p : P(x)}Q(x,p)$ is covariant. Note that
    \[
        (f_* (p, \phi(p)), \trans{f}{(p,\phi(p))}) : \sm{v :C(y)}\ho[C(f)]{(p, \phi(p))}{v}
    \]
    and similarly
    \[
    (((f_*(p),(f_* \circ \phi)(p))), (\trans{f}{p}, \trans{f}{\phi(p)})) : \sm{v :C(y)}\ho[C(f)]{(p, \phi(p))}{v}.
    \]
    The covariance of $C$ implies the type $\sm{v :C(y)}\ho[C(f)]{(p, \phi(p))}{v}$ is contractible. Hence,

    $$
    f_* (p, \phi(p)) = ((f_*(p),(f_* \circ \phi)(p)))
    $$
\end{proof}
    



Finally, we turn our attention to classifying universes for fibrations. In HoTT, the fact that all type families are fibrations allows us to view $\U$ as a classifying universe for fibrations. On the other hand, in sHoTT only some type families are covariant fibrations. While $\U$ is not a classifying universe for covariant fibrations, we can postulate the existence of a universe that is.
\begin{axiom}
    There is a Segal type $\uc$ along with objects $$\enn, \Unn, \voo : \uc,$$
    and decoding function
    $$\El : \uc \to \U$$
    such that\footnote{$\mathbb{N}^d$ is defined in 3.1} $\El(\enn) \equiv \mathbb{N}^d$, $\El(\Unn) \equiv \Un$, $\El(\voo) \equiv \vo$, and $\El(x)$ is discrete for all $x : \uc$.
    Moreover, for any map $B : A \to \uc$, it is the case that the family 
    $$\lam{x}\El (B(x)) : A  \to \U$$ is covariant. Moreover, if $A$ is Segal, then for any $x,y : A$ there is a term $$\overline{\ho[A]{x}{y}} : \uc$$ such that $$\El(\overline{\ho[A]{x}{y}}) \equiv \ho[A]{x}{y}.$$ Similarly, if $B$ is discrete, then for any $x,y : B$ there is a term $$\overline{x=_B y}:\uc$$ such that $$\El(\overline{x=_{B} y})\equiv x=_{B} y.$$ 
    Finally, for any $A : \uc$ and $P : \El(A) \to \uc$, there is a term
    $$\overline{\sm{a:A}\El(P(a))} : \uc$$ such that
    $$\El(\overline{\sm{a:A}\El(P(a))}) \equiv \sm{a:A}\El(P(a))$$
    

\end{axiom}



\end{document}