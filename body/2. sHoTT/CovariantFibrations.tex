\documentclass[main.tex]{subfiles}

\begin{document}
In section 2, we saw that all type families were fibrations and what that means specifically in HoTT. Of course in sHoTT type 
families are also fibrations, with respect to \textit{paths} that is. When discussing which type families lift simplicial shapes,
we must specify which shapes are being lifted. An important collection of type families are the ones that lift \textit{morphisms},
covariantly or contravariantly. Before we can precsiely define such type families, we must first make specify what it means to be a 
 "lifted morphism".
\begin{definition}[\cite{riehl_type_2017}, 8.1]
Given a type $A$, a type family $C : A \to \U$, points $x,y : A$, a morphism $f : \ho[A]{x}{y}$ and points $u : C(x)$ and $v : C(y)$, the type of \textbf{dependent morphisms over} $f$ \textbf{from} $u$ to $v$ is written as $$
\dho{C}{f}{u}{v} \defeq \exten{t : \osx}{C(f(t))}{\pardo}{[u,v]}.$$
\end{definition}
The dependent morphism represents the intuitive notion of "morphism in the total space". Alternatively, we can observe that a type family $C: A \to \U$ associates morphisms $f : \ho[A]{x}{y}$ to spans
% https://q.uiver.app/#q=WzAsMyxbMSwwLCJcXHNte3U6Qyh4KX17djpDKHkpfVxcZGhve0N9e2Z9e3V9e3Z9Il0sWzAsMSwiQyh4KSJdLFsyLDEsIkMoeSkiXSxbMCwxLCJcXGRvbSIsMl0sWzAsMiwiXFxjb2QiXV0=
\[\begin{tikzcd}
	& {\sm{u:C(x)}{v:C(y)}\dho{C}{f}{u}{v}} \\
	{C(x)} && {C(y)}
	\arrow["\dom"', from=1-2, to=2-1]
	\arrow["\cod", from=1-2, to=2-3]
\end{tikzcd}\]
We do not know a priori that $C$ depends functorially on $A$. We can instead restrict ourselves to the study of type families that have our
 desired property.
\begin{definition}[\cite{riehl_type_2017}, 8.2]
A type family is \textbf{covariant} if for every $f : \ho[A]{x}{y}$  and $u : C(x)$, the type $$\sm{v : C(y)} \dho{C}{f}{u}{v}$$
is contractible. When the context is clear, we write the specific inhabitant as $(\dtran{C}(f, u), \trans{f}{u})$ or 
even more tersely as $(f_*u, \trans{f}{u}).$

Dually\footnote{We will not explicitly state dual theorems and definitions when they are obvious from context.}, $C$ is \textbf{contravariant} if for every $f : \ho[A]{x}{y}$  and $v : C(y)$, the type $$\sm{v : C(x)} \dho{C}{f}{u}{v}$$
is contractible.
\end{definition}
As we stated before, we are interested in type families that "lift morphisms". The following lemma shows that covariant type families 
satisfies this property.
\begin{lemma}[\cite{riehl_type_2017}, 8.4]
    A type family $C : A \to \U$ is covariant if and only if for all morphisms $f : \ho[A]{x}{y}$ and points $u : C(x)$ the type $\exten{t : \osx}{C(f(t))}{0}{u}$ is contractible; there is a unique lifting of $f$ that starts at $u$.
\end{lemma}
\begin{remark}
    Covariant families are stable under subtitution. That is, for any $g : B \to A$ and covariant type family $C : A \to \U$, the type family
    $\lam{b}C(g(b))$ is also covariant.
\end{remark}
With these in hand, we can start stating facts that we expect from our covariant type families. To start, in HoTT, it is the case that
transporting along a type family is the same thing as applying a type family to a path, then coercing along the new path. A similar 
statement holds.
\begin{lemma}[\cite{riehl_type_2017}, 8.1]
    \label{lem:tpthomiscoethenapp}
    Let $C : B \to \U$ be covariant and $h : A \to B$. For any $f : \ho[A]{x}{y}$ and $u : C(h(x))$ we have that
    \[
         \dtran{C\circ h}(f, u) = \dtran{C}(\mapfunc{h}{f}, u)
    \]
\end{lemma}
\begin{proof}
    Note that 
    \[
        (\dtran{C\circ h}(f, u), \trans{f}{u}) : \sm{v : C(h(y))}\ho[(C \circ h)(f)]{u}{v} \equiv 
    \]
    and 
    \begin{align*}
        (\dtran{C}(\mapfunc{h}{f}, u), \trans{\mapfunc{h}{f}}{u}) : &\sm{v : C(h(y))}\ho[C(\mapfunc{h}{f})]{u}{v} \equiv \\
        & \sm{v : C(h(y))}\ho[C(h(f))]{u}{v}
    \end{align*}
    Since $C \circ h$ is covariant, the type $\sm{v : C(h(y))}\ho[C(h(f))]{u}{v}$ is contractible, 
    hence $\dtran{C\circ h}(f, u) = \dtran{C}(\mapfunc{h}{f}, u)$, as desired.
\end{proof}

We can also show that dependent morphisms in a covariant type family are equivalent to a path whose domain is a transport, analogous
 to the situation in HoTT.

\begin{lemma}[\cite{riehl_type_2017}, 8.15]
    \label{lem:depmorispath}
    If $C : A \to \U$ is covariant, $f : \ho[A]{x}{y}$, $u: C(x)$, and $v : C(y)$, then
    $$\ho[C(f)]{u}{v} \simeq (f_*u =_{C(y)}v).$$
\end{lemma}

Often times, we are in a situation where we are given a covariant family over a Segal base. The covariant family should faithfully
represent the compositions in the base, as it is functorial in the base. So, we should expect the total space of the family to be 
Segal as well, since the total space is composed of lifted morphisms.

\begin{lemma}[\cite{riehl_type_2017}, 8.8]
    If $A$ is segal and $C : A \to \U$ is covariant, then 
    $$\sm{a : A}C(a)$$
    is Segal.
\end{lemma}

For such covariant families over Segal bases, we can also show that "directed transporting" is functorial:
\begin{lemma}[\cite{riehl_type_2017}, 8.16]
    For any segal type $A$, covariant family $C : A \to \U$, $f : \ho[A]{x}{y}$, $g : \ho[A]{y}{z}$, and $u : C(X)$, we have
    \begin{gather*}
        g_*(f_*\,u) = (g\cdot f)_*\,u \qquad \normalfont \text{ and } \qquad (\idhom{x})_*\,u = u
    \end{gather*}
\end{lemma}


There are certain type families that are automatically covariant over a Segal base. For example, in category theory, we can make
 extensive use of the representable $\lam{x} \ho[A]{a}{x}$. In sHoTT, the story is the exact same. One useful property of the
  representable in sHoTT is that it allows us to characterize Segal types.
\begin{lemma}[\cite{riehl_type_2017}, 8.13]
    Given a type $A$ and a point $a : A$, then
    $$\lam{x} \ho[A]{a}{x} : A \to \uc$$  if and only if $A$ is Segal.
\end{lemma}

Lastly, covariant types over Segal bases have discrete fibers.

\begin{lemma}[\cite{riehl_type_2017}, 8.18]
    For any Segal type $A$ and covariant map $C : A \to \U$, and $x : A$, the type $C(x)$ is discrete.
\end{lemma}

\begin{corollary}[\cite{riehl_type_2017}, 8.19]
    If $A$ is Segal, then for any $x,y: A$ the type $\ho[A]{x}{y}$ is discrete.
\end{corollary}

\begin{corollary}[\cite{riehl_type_2017}, 8.20]
    If $A$ is discrete, then for any $x,y : A$, so is $x =_A y$.
\end{corollary}



Covariant type families act functorially in other important ways. For example, naturality comes for free, for maps between covariant
 types.

\begin{lemma}[\cite{riehl_type_2017}, 8.17]
    For any two covariant families $C,D : A \to \U$ and a fiberwise map $\phi : \prd{x : A}C(x) \to D(x)$, $f: \ho[A]{x}{y}$ and $u:C(x)$,
    $$\phi_y(f_*\,u) = f_*(\phi_x(u)).$$
\end{lemma}

We also enjoy some closure properties with covariant type families. The covariance of dependent functions passes over to dependent pairs:
\begin{lemma}
    \label{lem:covdomcovcodiscov}
    Given any type family $P : A \to \U$ and a covariant type family $Q : \prd{x : A}P(x) \to \U$,
    \[
    C \defeq \lam{(x,p)}Q(x,p) : (\sm{x:A}P(x)) \to \U
    \]
    is covariant.
\end{lemma}
\begin{proof}
    For any $f : \ho[\sm{x:A}P(x)]{(x,p)}{(y,p')}$ and $u : Q(x,p)$, we note that the type
    $$\exten{\preshape{t}{I}{\osx}}{Q(\fst(f(t)), \snd(f(t)))}{0}{u}$$
    is contractible by the covariance of $Q$. Hence $C$ is covariant.
\end{proof}

We can show that the path types in covariant families are covariant:
\begin{lemma}[\cite{riehl_type_2017}, 8.26]
    \label{lem:pathincoviscov}
    Suppose $C : A \to \U$ is covariant. Then 
    \[
    \lam{a}{u}{u}u=v : \prd{a:A}(C(a) \to C(a) \to \U)
    \]
    is covariant.
\end{lemma} 
The previous lemma is slightly easier to use in the following incarnation:
\begin{corollary}
    \label{cor:pathwithfunciscov}
    Suppose $C : A \to \U$ is covariant and we have maps $u : \prd{x:A}C(v)$ and $v : \prd{x:A}C(x)$ Then 
    \[
    \lam{x}u(x)=v(x) :(A \to  \U)
    \]
    is covariant.
\end{corollary} 
\begin{proof}
    Since $C$ is covariant, 
    \[\lam{a}{u}{u}u=v : \prd{a:A}(C(a) \to C(a) \to \U)\] is covariant by lemma~\ref{lem:pathincoviscov}.
    So, by lemma~\ref{lem:covdomcovcodiscov}, 
    \[
    \lam{(x,u,v)}u=v : (\sm{x:A}C(x)\times C(x))\to \U
    \] is covariant.
    We can then precompose with the map $x \mapsto (x,u(x),v(x))$ to form the family 
    \[
        \lam{x}u(x)=v(x),
    \]
    with covariance of this family coming from our remark above.
\end{proof}
Mapping into a covariant family is covariant: 
\begin{lemma}[\cite{riehl_type_2017}, 8.30]
    \label{lem:covcodiscov}
    For types $C$, $A$, and covariant family $B : C \to \U$, the type family $$\lam{x : C} A \to B(x) : C \to \U$$
    is also covariant.
\end{lemma}
Mapping out of a covariant family \textit{into a discrete tpye} is contravariant:
\begin{lemma}[\cite{riehl_type_2017}, 8.31]
    \label{lem:covdomiscon}
    For a discrete type $Y$ and a covariant type family $C : A \to \U$, the type
    $$\lam{x : A}C(a) \to Y$$
    is contravariant.
\end{lemma}


The previous lemma also implies that transporting a pair is the same thing as a pair of transports.
\begin{corollary}
    \label{cor:tptinsigma}
     Given type families $P : A \to \U$, $Q : \prd{x : A}P(x) \to \U$ that are covariant, a morphism $f : \ho[A]{x}{y}$, point $p : P(x)$, and map 
     $\phi : \prd{p : P(x)}Q(x,p)$,
    $$(f_*(p),(f_* \circ \phi)(p)) = f_* (p, \phi(p))$$
\end{corollary}
\begin{proof}
    By the previous lemma, $C \defeq \lam{x}\sm{p : P(x)}Q(x,p)$ is covariant. Note that
    \[
        (f_* (p, \phi(p)), \trans{f}{(p,\phi(p))}) : \sm{v :C(y)}\ho[C(f)]{(p, \phi(p))}{v}
    \]
    and similarly
    \[
    (((f_*(p),(f_* \circ \phi)(p))), (\trans{f}{p}, \trans{f}{\phi(p)})) : \sm{v :C(y)}\ho[C(f)]{(p, \phi(p))}{v}.
    \]
    The covariance of $C$ implies the type $\sm{v :C(y)}\ho[C(f)]{(p, \phi(p))}{v}$ is contractible. Hence,

    $$
    f_* (p, \phi(p)) = ((f_*(p),(f_* \circ \phi)(p)))
    $$
\end{proof}
    



As stated before, only some type families are fibrations. This means that in base sHoTT, unless we can cleverly apply closure properties,
we must prove covariance by hand. Similarly, we must carry data that a type family is covaraint everywhere we need it. We can do slightly better
We can instead axiomatize some classifier universe for covariant fibrations and enforce that any function landing in this universe is 
a covariant function.
\begin{axiom}
    There is a Segal universe $\uc$ along with objects $$\bar{\mathbb{N}}, \bar{\Un} : \uc$$
    and decoding function
    $$\El : \uc \to \U$$
    such that $\El(\bar{\mathbb{N}}) = \mathbb{N}$ and $\El(\bar{\Un}) = \Un$.
    Moreover, for any type family $B : A \to \uc$, it is the case that the family 
    $$\lam{x}\El (B(x)) : A  \to \U$$ is covariant. We often leave the coersion implicit.
\end{axiom}
\begin{remark}
    Fibers of covariant types are discrete, so discrete types are the points of $\uc$. This is especially reasonable if you observe
    that discrete types over $\Un$ are covariant. 
\end{remark}


\end{document}