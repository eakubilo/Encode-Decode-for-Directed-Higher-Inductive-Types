\documentclass[main.tex]{subfiles}

\begin{document}
Having read section \todo{do a section about n-types}, with the tools developed in this section
a natural question to ask is: ``can we generalize the notion of an n-type to sHoTT?" This question can be interpreted in many ways. There is
a lot of freedom that is introduced, as we have free pick of which paths to change, and whether we want to change them to an isomorphism or morphism.
For some types, all of these notions coincide. For example, the notion of set is one where parallel paths between points are equal. We can generalize this
notion and instead ask for a type where parallel morphisms between points are equal. That is, we can ask for a type to "be a relation".
\begin{definition}
A type $A$ is a relation if for all $x,y : A$ and $f,g : \ho[A]{x}{y}$, we have $f = g$.
\end{definition}

As a first example of this, we can show that Unit is a relation.

\begin{lemma}
    Unit is a relation.
\end{lemma}
\begin{proof}
    It suffices to show $\ho[\Un]{\star}{\star}$ is contractible. This is immediate as
    \begin{equation}
        \ndexten{\osx}{\Un}{\pardo}{[\star,\star]}
    \end{equation}
    has contractible fibers, thus the entire extension type is contractible by relative
    function extensionality.
\end{proof}
Natural numbers are also a relation
\begin{lemma}
$\N$ is a relation.
\end{lemma}
\begin{proof}
    Now, we define $\eco : \prd{n : \N}{m : \N} \ho[\N]{n}{m} \to \co\,n\,m$ with the defining equations
    \begin{align*}
        &\eco\,0\,0\,\_ \defeq \star&\\
        &\eco\,S(n)\,S(m)\,f\defeq \eco\,n\,m\,\phi(f)&
    \end{align*}

    and we define $\dco : \prd{n:\N}{m:\N}\co\,n\,m \to \ho[\N]{n}{m}$ as
    \begin{align*}
        &\dco\,0\,0\,\_ \defeq \idhom{0}&\\
        &\dco\,S(n)\,S(m)\,u\defeq \lam{i}S(\dco\, n\, m\, u)&
    \end{align*}
    Before we can show that these maps form a pair in a bi invertible equivalence, we define a map $r : \prd{n : \N}{f:\ho[\N]{n}{n}}f=\idhom{n}$
    Finally, we can show that these functions form a bi invertible equivalence by defining a map  
    \begin{align*}
        \eta \, 0 \, 0 \, \_ 
    \end{align*}
\end{proof}
\end{document}