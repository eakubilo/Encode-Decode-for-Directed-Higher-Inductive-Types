\documentclass[main.tex]{subfiles}
\begin{document}
With the concept of covariant fibrations developed, we are now equipped to delve into a foundational result in category theory: the Yoneda Lemma. The structure of covariant families naturally provides mappings necessary to state the Yoneda Lemma in sHoTT. That is, for any covariant $C : A \to \U$ and $a : A$, there are canonical maps
\begin{align*}
    \evid{C}{a} \defeq \lam{\phi}\phi(a, \idhom{a}) & : \left ( \prd{x : A}(\ho[A]{a}{x}) \to C(x) \right ) \to C(a)\\
    \yon{C}{a} \defeq  \lam{\phi}\phi(a, \idhom{a}) & : C(a) \to \left ( \prd{x : A}(\ho[A]{a}{x}) \to C(x) \right )
\end{align*}
For covariant families defined over Segal types, the Yoneda Lemma is fully realized as follows:
\begin{theorem}[\cite{riehl_type_2017}, 9.1]
    \label{thm:yonlemma}
    Given a covariant family $C : A \to \U$ over a Segal type $A$, for any $a : A$ the maps $\evid{C}{A}$ and $\yon{C}{a}$ form an equivalence.
\end{theorem}
We can also state a dependent version of the Yoneda lemma, which is analogous to a directed version of the path induction principle.

\begin{theorem}[\cite{riehl_type_2017}, 9.5]
    \label{thm:depyonlemma}
    For any Segal type $A$, $a : A$, and covariant type family $C : \prd{x:A}(\ho[A]{a}{x} \to \U)$, the function
    $$\evid{C}{a} \lam{\phi}\phi(a, \idhom{a}) : \left ( \prd{x : A}{f:\ho[A]{a}{x}}\to C(x,f) \right ) \to C(a, \idhom{a})$$
    is an equivalence.
\end{theorem}

\end{document}