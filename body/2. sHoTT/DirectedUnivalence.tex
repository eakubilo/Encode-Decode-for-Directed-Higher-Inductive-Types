\documentclass[main.tex]{subfiles}

\begin{document}
For discrete types A and B, we seek to make an identification between morphisms $\ho[\uc]{A}{B}$ and functions $A \to B$. Just as we do so in informal HoTT, we can make such an identification by defining some canonical function:

\begin{lemma}
\label{lem:can_define_htf}
For any two discrete types $A$ and $B$, there is a function
\begin{equation}
    \htf : \ho[\uc]{A}{B} \to (A \to B).
\end{equation}

\end{lemma}
\begin{proof}
    We begin by observing that the identity function on $\uc$, \begin{equation}
    \label{eqn:zetaisucovid}
        \id{\uc} : \uc \to \uc
    \end{equation}
    is covariant, so by Lemma~\ref{lem:covcodiscov} the type family $$C \defeq \lam{B : \uc}A \to \id{\uc}(B) : \uc \to \U$$
    is also covariant. Note that \eqref{eqn:zetaisucovid} can be restated as $C \equiv \lam{B : \uc} A \to B$. Finally, we appeal to the Yoneda lemma to define $\htf$:
    $$\htf \defeq  \yon{C}{A}(\id{A})(B) \defeq \lam{f}f_*(\id{A}) : \ho[\uc]{A}{B} \to (A \to B) $$
\end{proof}
\begin{remark}
    Note that $\htf$ acts functorially, as for any $f : \ho[\uc]{A}{B}$ and $g : \ho[\uc]{B}{C}$,
    \begin{align*}
        \htf(g \cdot f) \equiv\\
        (g\cdot f)_*(\id{A}) =\\
        g_*(f_*(\id{A}))
    \end{align*}
with the second equality following from Proposition 8.16. 

\end{remark}
Just as HoTT does not come equipped with tools to show idtoeqv is an equivalence, sHoTT also does not come equipped with the tools to show $\htf$
 is an equivalence. Instead, we can postulate \textit{directed univalence} as an axiom.

\begin{axiom}[Directed Univalence]
    For any discrete types $A,B : \uc$, the function defined in Lemma~\ref{lem:can_define_htf} is an equivalence.
\end{axiom}

For discrete types $A$ and $B$, we can break the equivalence up into:
\begin{itemize}
    \item An introduction rule for $\ho[\uc]{A}{B}$ denoted $\dua$ for "directed univalence axiom":
    $$\dua : (A \to B) \to (\ho[\uc]{A}{B})$$
    \item An elimination rule
    $$\htf \defeq \yon{C}{A}(\id{A}) : \ho[\uc]{A}{B} \to (A \to B)$$
    \item The propositional computation rule
        $$\htf(\dua(f)) = f$$
    \item The propositional uniqueness rule, which states for any $f : \ho[\uc]{A} {B}$,
    $$f = \dua(\htf(f))$$
\end{itemize}
Directed univalence also allows us to  identify identity morphisms with identity functions and composition of morphisms in $\uc$ with composition of functions, stated tersely as
\begin{align*}
    \idhom{A} = \dua(\id{A})\\
    \dua{g}\cdot\dua{f} = \dua(g \circ f).
\end{align*}
The first equality from the fact that $\id{A} = (\idhom{A})_*(\id{A}) \equiv \htf(\idhom{A})$ by proposition 8.16. To show the second, if we define $p \defeq \dua(f)$ and $q \defeq \dua(g)$, then
    $$\dua(g \circ f) = \dua(\htf(p) \circ \htf(q)) = \dua(\htf(p \cdot q)) = p \cdot q$$

With directed univalence, we can show that dependent transport is really just coercing a morphism of types.
\begin{lemma}
    \label{lem:tptissubst}
    For any covariant type $P$ and morphism $f : \ho[A]{x}{y}$,
    \begin{equation*}
        \dua(f_*) = P(f)
    \end{equation*}
\end{lemma}
\begin{proof}
    By lemma~\ref{lem:tpthomiscoethenapp}, it follows that 
    \begin{align*}
        \dua(f_*) \defeq \\
        \dua(\dtran{P}(f, -)) \simeq \\
        \dua(\dtran{\lam{x}x}(\mapfunc{P}f, -)) \simeq \\
        \dua(\htf(\mapfunc{P}f)) \simeq\\
        \mapfunc{P}f \defeq \\
        P(f)
    \end{align*}
\end{proof}
We can also show that a morphism between functions is a commutative square.
\begin{lemma}
    For any two covariant families $F, G : A \to \U$, a morphism $h : \ho[A]{x}{y}$ and maps $f : F(x) \to G(x)$, and $g : F(y) \to G(y)$ we have the following:
    \begin{equation*}
        \ho[(\lam{x}F(x)\to G(x))h]{f}{q} \simeq g \circ h_* = h_*  \circ f
    \end{equation*}
\end{lemma}
\begin{proof}
    By directed univalence,
    \begin{equation*}
        \ho[(\lam{x}F(x)\to G(x))h]{f}{q} \simeq
        \ho[{(\lam{x}\ho[\uc]{F(x)}{G(x)})h}]{\dua f}{\dua g}
    \end{equation*}
    By lemma~\ref{lem:compissquare}, this is equivalent to
    \begin{equation*}
        \dua g \cdot F(h) = G(h) \cdot \dua f
    \end{equation*}
    which transports to
    \begin{equation*}
        \dua g \cdot \dua h_* = \dua h_* \cdot \dua f
    \end{equation*}
    by lemma~\ref{lem:tptissubst}. Lemma~\ref{lem:equivissurjemb} and the functorality of $\htf$ allow us to push
    this equivalence to the type
    \begin{equation*}
        g \circ h_* = h_* \circ f
    \end{equation*}
\end{proof}
Directed univalence also extends to characterizing dependent functions as morphisms in $\uc$.
\begin{lemma}
    \label{lem:depfuncishom}
    For any type family $P : A \to \uc$, for all $x : A$ and $Q : \tilde{P} \to \uc$, there is an equivalence

    $$\prd{p : P(x)} Q(x,p) \simeq \sm{f : \ho[\uc]{P(x)}{\sm{p : P(x)} Q(x,p)}} \dua(\fst) \circ f = \idhom{P(x)}, $$
    where $\fst : \sm{p : P(x)}Q(x,p) \to P(x)$
\end{lemma}
\begin{proof}
    We begin by noting the following equivalences
    \begin{align*}
        \sm{f : \ho[\uc]{P(x)}{\sm{p : P(x)} Q(x,p)}} \dua(\fst) \circ f = \idhom{P(x)} \simeq \\
        \sm{f : P(x)\to \sm{p : P(x)} Q(x,p)}\dua(\fst) \circ \dua(f) = \idhom{P(x)}\simeq \\
        \sm{f : P(x)\to \sm{p : P(x)} Q(x,p)}\dua(\fst \circ f) = \idhom{P(x)}\simeq \\
        \sm{f : P(x)\to \sm{p : P(x)} Q(x,p)} \fst \circ f = \id{P(x)}
    \end{align*}
    with the first equivalence following from lemma~\ref{lem:equivbaseequivtot}. The second equivalence is a result of the functorality 
    of $\dua$ and lemmas~\ref{lem:compisequiv} and~\ref{lem:equivistotequiv}. The third equivalence comes from~\ref{lem:equivissurjemb}. Now that
    we are working with functions, we can reduce the types further:
    \begin{align*}
       \sm{f : P(x) \to \sm{p : P(x)}Q(x,p)}\fst \circ f = \id{P(x)} \simeq \\
       \sm{(f_1,f_2) : \sm{g : P(x) \to P(x)}\prd{p : P(x)}Q(x,g(p))}f_1 = \id{P(x)} \simeq\\
       \sm{f : P(x) \to P(x)}(\prd{p : P(x)}Q(x,f(p))) \times (f = \id{P(x)}) \simeq \\
       \sm{f : P(x) \to P(x)} (f = \id{P(x)})\times(\prd{p : P(x)}Q(x,f(p))) \simeq \\
       \sm{(f, p): \sm{g : P(x) \to P(x)} g = \id{P(x)}} \prd{p : P(x)}Q(x,f(p)) \simeq \\
       \prd{p : P(x)}Q(x,p)
    \end{align*}
    with the first equivalence following from lemma~\ref{lem:ttaoc}. The second follows from
    lemma~\ref{lem:sigisassoc}. The third follows from lemma~\ref{lem:sigissymm}. The fourth follows again from
    lemma~\ref{lem:sigisassoc}. The final equivalence follows from lemma~\ref{lem:contractiblesingletons} and lemma~\ref{lem:contractibleissimple}.
     Thus, we can get our desired equivalence by composing the two equivalences formed in this proof.
    \end{proof}

We can even show that commutativity is sufficient to get a map into a type.

\begin{lemma}
    \label{lem:commisfunarr}
    For any covariant type families $P : A \to \U$, $Q : \prd{x : A}P(x) \to \U$, morphisms $f : \ho[A]{x}{z}$ and dependent functions
    $\phi : \prd{p : P(x)}Q(x,p)$, $\psi : \prd{p : P(z)}Q(z,p)$, there is a map 
    \begin{equation}
        \label{eqn:commisfunarr}
        f_* \circ \phi = \psi \circ f_* \to \ho[(\lam{x}\prd{p : P(x)}Q(x,p))f]{\phi}{ \psi}
    \end{equation}
\end{lemma}
\begin{proof}
    We start by defining a map
    $$\zeta : \prd{p : P(x)}Q(x,p) \to (P(x) \to \sm{p : P(x)}Q(x,p))$$
    with the defining equation
    $$\zeta \defeq \lam{g}{p}(p, g(p)).$$
    The codomain of \ref{eqn:commisfunarr} can be written as a morphism of sections:
    \begin{equation*}
        \ho[(\lam{x}(\sm{h : P(x) \to \sm{p : P(x)} Q(x,p)} \fst \circ h = \id{P(x)}))f]{(\zeta(\phi), \refl{})}{(\zeta(\psi), \refl{})}
    \end{equation*}
    by lemma~\ref{lem:depfuncishom}. We can then write this type as its extension type as follows:
    \begin{equation*}
        \exten{t : \osx}{\sm{h : P(f(t)) \to \sm{p : P(f(t))}Q(f(t), p)}\fst \circ h = \id{P(f(t))}}{\pardo}{[(\zeta(\phi), \refl{}), (\zeta(\psi), \refl{})]}
    \end{equation*}
    Noting that this type is a "product of sigmas", we can use the type theoretical axiom of choice, lemma ~\ref{lem:taocwet}, to write it as a "sigma of products":
    \begin{equation}
        \label{eqn:commisfunarr1}
        \sm{(h : \ho[(\lam{x} P(x) \to \sm{p : P(x)}Q(x,p)))f]{\zeta(\phi)}{\zeta(\psi)}}\exten{t : \osx}{\fst \circ h(t) = \id{P(f(t))}}{\pardo}{[\refl{}, \refl{}]}.
    \end{equation}
    We aim to inhabit~(\ref{eqn:commisfunarr1}). Let $h : f_* \circ \phi = \psi \circ f_*$. For any $p : P(x)$, we can construct a term
    $$h \defeq (\refl{}, \happly(h)(p)) : (f_*(p), (f_* \circ \phi)(p) ) = (f_*(p), \psi( f_*(p)))$$
    Note that $(f_*(p), \psi( f_*(p))) \equiv (\lam{p}(p, \psi(p)))(f_*(p))$. In conjunction with lemma~\ref{cor:tptinsigma}, there is a term
    $$h' : f_* \circ (\lam{p}(p, \phi(p)))  = (\lam{p}(p, \psi(p)))\circ f_*, $$ 
    which by lemma~\ref{lem:compissquare} induces a term
    $$\bar{h} : \ho[(\lam{x} P(x) \to \sm{p : P(x)}Q(x,p))f]{\zeta(\phi)}{\zeta(\psi)},$$ noting the covariance of
    $\lam{x}\sm{p:P(x)}Q(x,p)$ by lemma~\ref{lem:covdomcovcodiscov}, that is represented by the diagram
% https://q.uiver.app/#q=WzAsNCxbMCwwLCJQKHgpIl0sWzAsMiwiUCh5KSJdLFszLDAsIlxcc217cDpQKHgpfVEoeCxwKSJdLFszLDIsIlxcc217cDpQKHkpfVEoeSxwKSJdLFswLDIsIlxcemV0YShcXHBoaSkiXSxbMSwzLCJcXHpldGEoXFxwc2kpIl0sWzAsMSwiUChmKSIsMV0sWzIsMywiXFxzbXtwOlAoZil9UShmLHApIiwxXV0=
\[\begin{tikzcd}
	{P(x)} &&& {\sm{p:P(x)}Q(x,p)} \\
	\\
	{P(y)} &&& {\sm{p:P(y)}Q(y,p)}
	\arrow["{\zeta(\phi)}", from=1-1, to=1-4]
	\arrow["{P(f)}"{description}, from=1-1, to=3-1]
	\arrow["{\sm{p:P(f)}Q(f,p)}"{description}, from=1-4, to=3-4]
	\arrow["{\zeta(\psi)}", from=3-1, to=3-4]
\end{tikzcd}\]
    Note here that the horizontal represents a type theoretic function, while the verticals represent simplicial morphisms.
    We can pointwise compose $\bar{h}$ with the term $\lam{t}{p}\fst(p):\ho[(\lam{x} (\sm{p : P(x)}Q(x,p)) \to P(x))f]{\fst}{\fst}$
    which is represented by the diagram
% https://q.uiver.app/#q=WzAsNCxbMCwwLCJcXHNte3A6UCh4KX1RKHgscCkiXSxbMCwyLCJcXHNte3A6UCh5KX1RKHkscCkiXSxbMiwwLCJQKHgpIl0sWzIsMiwiUCh5KSJdLFswLDEsIlxcc217cDpQKGYpfVEoZixwKSIsMV0sWzAsMiwiXFxmc3QiLDFdLFsxLDMsIlxcZnN0IiwxXSxbMiwzLCJQKGYpIiwxXV0=
\[\begin{tikzcd}
	{\sm{p:P(x)}Q(x,p)} && {P(x)} \\
	\\
	{\sm{p:P(y)}Q(y,p)} && {P(y)}
	\arrow["\fst"{description}, from=1-1, to=1-3]
	\arrow["{\sm{p:P(f)}Q(f,p)}"{description}, from=1-1, to=3-1]
	\arrow["{P(f)}"{description}, from=1-3, to=3-3]
	\arrow["\fst"{description}, from=3-1, to=3-3]
\end{tikzcd}\]    
to yield a term 
    $$\kappa \defeq \lam{t}\fst \circ \bar{h}(t) : \ho[(\lam{x} P(x) \to P(x))f]{\id{P(x)}}{\id{P(y)}},$$ represented by the diagram
% https://q.uiver.app/#q=WzAsNCxbMCwwLCJQKHgpIl0sWzAsMiwiUCh5KSJdLFsyLDAsIlAoeCkiXSxbMiwyLCJQKHkpIl0sWzAsMSwiUChmKSIsMV0sWzAsMiwiXFxpZHtQKHgpfSIsMV0sWzIsMywiUChmKSIsMV0sWzEsMywiXFxpZHtQKHkpfSIsMV1d
\[\begin{tikzcd}
	{P(x)} && {P(x)} \\
	\\
	{P(y)} && {P(y)}
	\arrow["{\id{P(x)}}"{description}, from=1-1, to=1-3]
	\arrow["{P(f)}"{description}, from=1-1, to=3-1]
	\arrow["{P(f)}"{description}, from=1-3, to=3-3]
	\arrow["{\id{P(y)}}"{description}, from=3-1, to=3-3]
\end{tikzcd}\]
    Directed univalence allows us to turn this morphism-function amalgamation digram into a pure morphism diagram. That is, there is a term
    \begin{equation}
        \label{eqn:duacommisarr}
        \lam{t}\dua(\kappa(t)) : \ho[{(\lam{x} \ho[\uc]{P(x)}{P(x)})f}]{\idhom{P(x)}}{\idhom{P(y)}}
    \end{equation}
    represented by the diagram
    \[\begin{tikzcd}
	{P(x)} && {P(x)} \\
	\\
	{P(y)} && {P(y)}
	\arrow["{\idhom{P(x)}}"{description}, from=1-1, to=1-3]
	\arrow["{P(f)}"{description}, from=1-1, to=3-1]
	\arrow["{P(f)}"{description}, from=1-3, to=3-3]
	\arrow["{\idhom{P(y)}}"{description}, from=3-1, to=3-3]
\end{tikzcd}\]
    We also make note of the fact that there is an "identity term" that inhabits the same type as \ref{eqn:duacommisarr}:
    $$\lam{t}\idhom{P(f(t))} : \ho[{(\lam{x} \ho[\uc]{P(x)}{P(x)})f}]{\idhom{P(x)}}{\idhom{P(y)}}.$$
    Thus, by lemma \todo{show that squares with the same side are equal in a segal type}, there is an inhabitant of the type
    $$\lam{t}\dua(\kappa(t)) = \lam{t}\idhom{P(f(t))}.$$
    This is equivalently typed as 
    $$\kappa = \lam{t}\id{P(f(t))} : \exten{t:\osx}{ \fst \circ \bar{h}(t) = \id{P(f(t))}}{\pardo}{[\refl{}, \refl{}]}$$
    Or, by unraveling definitional equalities:
    $$\lam{t}\fst \circ \bar{h}(t) = \lam{t}\id{P(f(t))}.$$
    This induces an inhabitant of the type 
    $$ \exten{t:\osx}{ \fst \circ \bar{h}(t) = \id{P(f(t))}}{\pardo}{[\refl{}, \refl{}]}$$
    as desired.
\end{proof}


\end{document}