\documentclass[main.tex]{subfiles}

\begin{document}
Having established our three-layer type theory and extension types, we can start developing the core aspects of sHoTT. We initiate this development by utilizing our extension types to investigate the presence of simplicial structures within types.

\begin{definition}[\cite{riehl_type_2017}, 5.1]
    For a type $A$ and points $x,y : A$ which induce a term $[x,y]$\footnote{This is shorthand for $\recOr{t \equiv 0, t \equiv 1}{x}{y}$} in context $\pardo$, we write
    $$\ho[A]{x}{y} \defeq \ndexten{\osx}{A}{\pardo}{[x,y]}$$
    and say that an element of $\ho[A]{x}{y}$ is a \textbf{morphism} from $x$ to $y$. 
\end{definition}

Before we probe for more shapes, we turn our attention to the functorality of functions in sHoTT. In the last section, we saw that functions acted functorially on paths using path induction. In sHoTT, functions act functorially on morphisms by definition.
Consequently, we have the following lemmas.
\begin{lemma}[\cite{riehl_type_2017}, 6.1]
    Let $A,B$ be types and $f : A \to B$. Then, given $x,y : A$, there is a function 
    \[\mapfunc{f} : \ho[A]{x}{y} \to \ho[B]{f x}{f y}\]
    When the context is clear, we often write $f(h)$ rather than $\mapfunc{f}h$ for $h : \ho[A]{x}{y}$.
\end{lemma}
Dependent functions can also be definitionally applied to morphisms.
\begin{lemma}
    Let $B : A \to \U$ and $\phi : \prd{x : A}B(x)$. Then, given $x,y : A$, there is a function
    $$\mapdepfunc{\phi} : \prd{f : \ho[A]{x}{y}}\ho[B(f)]{\phi(x)}{\phi(y)}$$
\end{lemma}
Our exploration extends beyond one dimension. We can also probe for $2$-dimensional shapes and higher. For the purposes of this thesis, we focus on $1$ and $2$-dimensional shapes, and we will not delve into higher dimensional definitions here.
\begin{definition}[\cite{riehl_type_2017}, 5.2]
    For points $x,y,z : A$ and morphisms $f : \ho[A]{x}{y}$, $g : \ho[A]{y}{z}$, and $h : \ho[A]{x}{z}$ which induces a term $[x,y,z,f,g,h]$ in context $\pardt$, we write
    $$\homtwo{A}(x,y,z,f,g,h) \defeq \ndexten{\tsx}{A}{\pardt}{[x,y,z,f,g,h]}$$
    or, in an abbreviated fashion,
    $$\homtwoshort{A}(x,y,z,f,g,h)$$
    and say that an element of $\homtwoshort{A}(x,y,z,f,g,h)$ is a $2$-\textbf{morphism}.
\end{definition}
\begin{definition}
    For points $a,b,c,d : A$, and morphisms $f : \ho[A]{a}{d}$, $g : \ho[A]{b}{c}$, $h : \ho[A]{a}{b}$, and $k : \ho[A]{d}{c}$ which induces a term \linebreak $[a,b,c,d,f,k,h,g]$ in context $\pards$, we write
    $$\diag{A}(a,b,c,d,f,k,h,g) \defeq \ndexten{\mathbbm{2}\times\mathbbm{2}}{A}{\pards}{[a,b,c,d,f,k,h,g]}$$
    or, in an abbreviated fashion,
    $$\diagshort{A}(a,b,c,d,f,k,h,g)$$
    and say that an element of $\diagshort{A}(a,b,c,d,f,k,h,g)$ is a \textbf{diagram}.
\end{definition}
Importantly, $\mathbbm{2} \times \mathbbm{2}$ is the pushout of two copies of $\tsx$ along their common boundary. We can use this fact in two ways. The first allows us to create ``connection squares'', which are simple but powerful diagrams:
\begin{lemma}[\cite{riehl_type_2017}, 3.5]
    For any type $A$, points $a,b:A$, and morphism $f : \ho[A]{a}{b}$, there are diagrams
    $$\bigvee\nolimits_f : \diag{A}(x,y,y,y,f,\idhom{y},f,\idhom{y})$$
    $$\bigwedge\nolimits_f : \diag{A}(x,x,y,x,\idhom{x},f,\idhom{x},f)$$
\end{lemma}

In the second, we note that diagrams inherit the pushout property. More specifically, the following lemma shows that diagrams are a pushout of two $2$-morphisms.
\begin{lemma}
    \label{lem:squareistwotriangles}
    For any type $A$ and morphisms $f : \ho[A]{a}{d}$, $g : \ho[A]{b}{c}$, $h : \ho[A]{a}{b}$, and $k : \ho[A]{d}{c}$, there is an equivalence 
    \begin{equation}
        \label{eqn:squareistwotriangles}
        \diagshort{A}(a,b,c,d,f,k,h,g) \simeq \sm{p : \ho[A]{a}{c}}\homtwoshort{A}(a,d,c,f,k,p) \times \homtwoshort{A}(a,b,c,h,g,p) 
    \end{equation} 
\end{lemma}
\begin{proof}
    From left to right, 
    \begin{equation*}
        \alpha \mapsto \left (\lam{t : \mathbbm{2}}\alpha(t,t),\lam{\langle t,s \rangle : \tsx} \alpha(s,t), \lam{\langle t,s \rangle : \tsx} \alpha(t,s) \right ).
    \end{equation*}
    
    From right to left, 
    \begin{equation*}
        (p, \alpha_1, \alpha_2) \mapsto \lam{\left \langle t,s \right \rangle : \mathbbm{2} \times \mathbbm{2}}\recOr{t\leq s, s \leq t }{\alpha_1(s, t)}{\alpha_2(t,s)}.
    \end{equation*}

    We first check, in the case of the map from the right to the left, that the two provided terms for $\mathsf{rec}_{\lor}$ agree along the common boundary. Since
    $\alpha_1(t,t) \equiv \alpha_2(t,t) \equiv p$ for all $t : \mathbbm{2}$, the map is well-formed.
    
    The round trip from left, to right, then back to the left takes a term $$\alpha : \ndexten{\mathbbm{2}\times\mathbbm{2}}{A}{\partial \square}{[f,k,h,g]}$$
    to 
    $$
    \left (\lam{t : \mathbbm{2}}\alpha(t,t),\lam{\langle t,s \rangle : \tsx} \alpha(s,t), \lam{\langle t,s \rangle : \tsx} \alpha(t,s) \right )
    $$
    then finally to
    $$\lam{\langle t,s \rangle : \mathbbm{2} \times \mathbbm{2}}\recOr{t\leq s, s \leq t }{\alpha(t, s)}{\alpha(t,s)}$$
    which is definitionally equal to $\alpha$.

    The round starting from the right takes the term
    $$(p, \alpha_1, \alpha_2): \sm{p : \ho[A]{a}{c}}\homtwoshort{A}(a,d,c,f,k,p) \times \homtwoshort{A}(a,b,c,h,g,p)$$ to 
    $$ \lam{\left \langle t,s \right \rangle : \mathbbm{2} \times \mathbbm{2}}\recOr{t\leq s, s \leq t }{\alpha_1(s, t)}{\alpha_2(t,s)}.$$
    We can evaluate what the map from left to right does on this term coordinate-wise. In the first coordinate, we have 
    $$\lam{t:\mathbbm{2}}(\lam{\left \langle t,s \right \rangle : \mathbbm{2} \times \mathbbm{2}}\recOr{t\leq s, s \leq t }{\alpha_1(s, t)}{\alpha_2(t,s)})(t,t) \equiv p.$$
    In the second coordinate, we have the following term in the context of $\langle t,s \rangle : \tsx$:
    \begin{equation} 
        \label{eqn:scndcoordinate}
        (\lam{\left \langle t,s \right \rangle : \mathbbm{2} \times \mathbbm{2}}\recOr{t\leq s, s \leq t }{\alpha_1(s, t)}{\alpha_2(t,s)})(s,t)
    \end{equation}
    Noting that $s \leq t$, equation \ref{eqn:scndcoordinate} computes to
    $\alpha_1(t,s)$. Thus, the second coordinate is simply $\alpha_1$.

    In the third coordinate, we have the following term in the context of $\langle t,s \rangle : \tsx$:
    \begin{equation} 
        \label{eqn:thrdcoordinate}
        (\lam{\left \langle t,s \right \rangle : \mathbbm{2} \times \mathbbm{2}}\recOr{t\leq s, s \leq t }{\alpha_1(s, t)}{\alpha_2(t,s)})(t,s)
    \end{equation}
    Again, since $s \leq t$, equation \ref{eqn:thrdcoordinate} computes to $\alpha_2(t,s)$, thus the third coordinate is simply $\alpha_2$.

    Since both round trips are definitionally the identity, the equivalence in \ref{eqn:squareistwotriangles} holds.
\end{proof}
Observe that, although we have introduced categorical terms such as diagrams and morphisms, a formal definition of a category in sHoTT is still forthcoming. Recall that in HoTT, path composition was a natural consequence of path induction, subsequently revealing the groupoidal nature of types. While it is tempting to anticipate a similar derivation for morphism composition, this expectation does not hold. Morphism composition in sHoTT requires a different approach, as it is not a direct result of inductive principles as with paths.

This distinction highlights a core principle of sHoTT: that types exhibit the structure of infinity categories. To illustrate this concept through an analogy with programming languages, path composition mirrors the behavior of a parametrically polymorphic function, where the operation is uniform across all types. In contrast, morphism composition is analogous to type class polymorphism, where the implementation of composition can vary depending on the specific type.

It is not the case that every type naturally exhibits the structure of a category. Rather, categorical structure emerges specifically in those types that admit a well-defined composition operation. Indeed, we can demonstrate that the existence of such a composition operation serves as the foundational property from which all other aspects of categorical structure are derived. Consequently, we proceed to our next definition.
\begin{definition}[\cite{riehl_type_2017}, 5.3]
    \label{def:def3.3}
    Let $A$ be a type, $x,y,z : A$ and $f : \ho[A]{x}{y}$ and $g : \ho[A]{y}{z}$. We say $A$ is \textbf{Segal} if the type $${\sm{h : \ho[A]{x}{z}} \hot[A]{f}{g}{h}} $$
    is contractible. The unique inhabitant is denoted by $(g\cdot f, \comp{g}{f}).$
\end{definition}

Thus, in Segal types, composable morphisms directly correspond to 2-mor{\linebreak}phisms. Before detailing the categorical structure of Segal types, we will examine alternative characterizations and key lemmas. We begin with an alternative characterization.

\begin{lemma}[\cite{riehl_type_2017}, 5.5]
    \label{lem:segalisrestrict}
    A type $A$ is Segal if and only if the restriction map
    $$
    (\tsx \to A) \to (\tohorn \to A) 
    $$
    is an equivalence. 
    \end{lemma}
    In the proof of \ref{lem:segalisrestrict}, \cite{riehl_type_2017} proves along the way that there is an equivalence
    \begin{equation}
        \sm{h : \ho[A]{x}{z}}\homtwo{A}(x,y,z,f,g,h) \simeq \ndexten{\tsx}{A}{\tohorn}{[x,y,z,f,g]}.
    \end{equation}
    As we will see, it is also useful to consider summing over the other sides of a 2-morphism.
    \begin{corollary}
        In any type $A$, there are equivalences
        \begin{equation}
            \label{rem:sumoverfistt}
        \sm{f : \ho[A]{x}{y}}\homtwo{A}(x,y,z,f,g,h) \simeq \ndexten{\tsx}{A}{\tthorn}{[x,y,z,g,h]}
        \end{equation}
        and
        \begin{equation}
        \sm{g : \ho[A]{y}{z}}\homtwo{A}(x,y,z,f,g,h) \simeq \ndexten{\tsx}{A}{\tzhorn}{[x,y,z,f,h]}.
        \end{equation}
    \end{corollary}

Now that we've covered that digression, we can examine the categorical structure of Segal types. We already have three key components: objects (which are points of a type), morphisms (which are points of the morphism type), and a weak composition function that arises from the Segal definition. Unit and associativity laws also follow.

\begin{lemma}[\cite{riehl_type_2017}, 5.8]
    If $A$ is a Segal type with terms $x,y : A$, then for any $f : \ho[A]{x}{y}$ we have $\idhom{y}\circ f = f$ and $f \circ \idhom{x} = f$.
\end{lemma}

\begin{lemma}[\cite{riehl_type_2017}, 5.9]
    If $A$ is a Segal type with terms $x,y,z,w : A$, then for any $f : \ho[A]{x}{y}$, $g : \ho[A]{y}{z}$, $h : \ho[A]{z}{w}$ we have $(h \circ g )  \circ f = h \circ (g \circ f).$
\end{lemma}

We turn now to the analogue of homotopies in sHoTT. Given a Segal type $A$, we have two ways of saying two morphisms $f,g : \ho[A]{x}{y}$ are the same. We can either:
\begin{enumerate}
    \item Provide a path $f = g$
    \item Provide a 2-simplex $\homtwo{A}(x,x,y,\idhom{x},f,g)$
\end{enumerate}
These two perspectives turn out to be equivalent.
\begin{lemma}[\cite{riehl_type_2017}, 5.10]
    \label{lem:pathis2mor}
    For any $f,g : \ho[A]{x}{y}$ in a Segal type $A$, the natural map
    $$f = g \to \homtwo{A}(x,x,y, \idhom{x}, f,g)$$
    is an equivalence.
\end{lemma}
Furthermore, composition and equality are closely related.
\begin{lemma}[\cite{riehl_type_2017}, 5.12]
    \label{lem:compequalis2mor}
    For any $f: \ho[A]{x}{y}$, $g : \ho[A]{y}{z}$, and $h : \ho[A]{x}{z}$ in a Segal type $A$, the natural map
    $$g \cdot f = h \to \homtwo{A}(x,y,z,f,g,h)$$
    is an equivalence.
\end{lemma}
Extending this idea, commutative squares correspond to commuting morphisms.
\begin{lemma}
    \label{lem:compissquare}
    Given a Segal type $A$ and compatible morphisms $f : \ho[A]{a}{d}$, \linebreak $k:\ho[A]{d}{c}$, $h:\ho[A]{a}{b}$, and $g : \ho[A]{d}{c}$ then
    \begin{equation}
        \diag{A}(a,b,c,d,f,k,h,g) \simeq k\cdot f = g \cdot h
    \end{equation}
\end{lemma}
\begin{proof}
By lemma~\ref{lem:squareistwotriangles},
$$        \diagshort{A}(a,b,c,d,f,k,h,g) \simeq \sm{p : \ho[A]{a}{c}}\homtwoshort{A}(a,d,c,f,k,p) \times \homtwoshort{A}(a,b,c,h,g,p) .$$
By lemma~\ref{lem:compissquare}, the both $2$-morphism types in the codomain can be equated with path types
$$\sm{p : \ho[A]{a}{c}}\homtwoshort{A}(a,d,c,f,k,p) \times \homtwoshort{A}(a,b,c,h,g,p) \simeq \sm{p : \ho[A]{a}{c}} k\cdot f = p \times g \cdot h = p.$$
By lemma~\ref{lem:sigisassoc}, we can reassociate the pair type as follows
$$\sm{p : \ho[A]{a}{c}} k\cdot f = p \times g \cdot h = p \simeq \sm{t : \sm{p : \ho[A]{a}{c}} k\cdot f = p } g \cdot h = \fst(t).$$
Note that the base is contractible by singleton contractibility, with center of contraction $k \cdot f$. Thus, by lemma~\ref{lem:contractibleissimple}, the codomain is equivalent to 
$$g \cdot h = k \cdot f. $$ Our desired equation is the composition of all of the equivalences constructed in this proof.
\end{proof}
\end{document}