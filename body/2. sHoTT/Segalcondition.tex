\documentclass[main.tex]{subfiles}

\begin{document}
With our three-layer type theory and extension types in tow, we can start developing the core aspects of sHoTT. We start by using our extension types
to probe for simplicial shapes in types.

\begin{definition}[\cite{riehl_type_2017}, 5.1]
    For a type $A$ and points $x,y : A$ which induce a term $[x,y]$\footnote{Spelled out, this is a term that depends on $t : \pardo $ that evaluates to $x$ when $t \equiv 0$ and $y$ when $t \equiv 1$.} in context $\pardo$, we write
    $$\ho[A]{x}{y} \defeq \ndexten{\osx}{A}{\pardo}{[x,y]}$$
    and say that an element of $\ho[A]{x}{y}$ is a \textbf{morphism} from $x$ to $y$. 
\end{definition}
\begin{definition}[\cite{riehl_type_2017}, 5.2]
    For points $x,y,z : A$, and morphisms $f : \ho[A]{x}{y}$, $g : \ho[A]{y}{z}$, and $h : \ho[A]{x}{z}$ which induces a term $[x,y,z,f,g,h]$\footnote{Similarly, this is a term that depends on $\pair{t}{s} :\pardt$ that evaluates to:
    \begin{itemize}
        \item $x$ when $\pair{t}{s} \equiv \pair{0}{0}$ 
        \item $y$ when $\pair{t}{s} \equiv \pair{1}{0}$ 
        \item $z$ when $\pair{t}{s} \equiv \pair{1}{1}$
        \item $f$ when $\pair{t}{s} \equiv \pair{t}{0}$
        \item $g$ when $\pair{t}{s} \equiv \pair{1}{s}$
        \item $h$ when $\pair{t}{s} \equiv \pair{t}{t}$
    \end{itemize} } in context $\pardt$, we write
    $$\homtwo{A}(x,y,z,f,f,h) \defeq \ndexten{\tsx}{A}{\pardt}{[x,y,z,f,g,h]}$$
    or, in an abbreviated fashion,
    $$\homtwoshort{A}(x,y,z,f,g,h)$$
    and say that an element of $\homtwoshort{A}(x,y,z,f,g,h)$ is a $2$-\textbf{morphism}.
\end{definition}
One result that we'll make use of in this section states that $\mathbbm{2} \times \mathbbm{2}$ is the pushout of two copies of $\tsx$ along their common boundary.
\begin{lemma}
    For any type $A$ and morphisms $f : \ho[A]{a}{d}$, $g : \ho[A]{b}{c}$, $h : \ho[A]{a}{b}$, and $k : \ho[A]{d}{c}$, there is an equivalence 
    $$\ndexten{\mathbbm{2}\times\mathbbm{2}}{A}{\partial \square}{[f,k,h,g]} \simeq \sm{p : \ho[A]{a}{c}}\homtwoshort{A}(a,d,c,f,k,p) \times \homtwoshort{A}(a,b,c,h,g,p)  $$
\end{lemma}
\begin{proof}
    From left to right, $$\alpha \mapsto \left (\lam{t : \mathbbm{2}}\alpha(t,t),\lam{\langle t,s \rangle : \tsx} \alpha(t,s), \lam{\langle t,s \rangle : \tsx} \alpha(s,t) \right ).$$
    
    From right to left, $$(\_, \alpha_1, \alpha_2) \mapsto \lam{\langle t,s \rangle : \mathbbm{2} \times \mathbbm{2}}\recOr{t\leq s, s \leq t }{\alpha_1(t, s)}{\alpha_2(s,t)}.$$

    The round trip from left, to right, back to left takes a term $\alpha : \ndexten{\mathbbm{2}\times\mathbbm{2}}{A}{\partial \square}{[f,k,h,g]}$ to 
    $$\lam{\langle t,s \rangle : \mathbbm{2} \times \mathbbm{2}}\recOr{t\leq s, s \leq t }{\alpha(t, s)}{\alpha(t,s)}$$
    which is definitionally equal to $\alpha$.

    The round trip from right, to left, back takes a term $(p, \alpha_1, \alpha_2)$ to $$(p, )$$
    \todo{finish proof, take it from the start and make sure you get the indices right}
\end{proof}

In the last section, we saw that functions acted functorially on paths using path induction. In sHoTT, we get nicer behavior.
Functions definitionally act functorially on morphisms, edified by the following lemma. 
\begin{lemma}[\cite{riehl_type_2017}, 6.1]
    Let $A,B$ be types and $f : A \to B$. Then, given $x,y : A$, there is a function 
    \[\mapfunc{f} : \ho[A]{x}{y} \to \ho[B]{f x}{f y}\]
    When the context is clear, we often write $f(h)$ rather than $\mapfunc{f}h$ for $h : \ho[A]{x}{y}$.
\end{lemma}
The dependent version of the previous lemma follows.
\begin{lemma}
    Let $B : A \to \U$ and $\phi : \prd{x : A}B(x)$. Then, given $x,y : A$, there is a function
    $$\mapdepfunc{\phi} : \prd{f : \ho[A]{x}{y}}\ho[B(f)]{\phi(x)}{\phi(y)}$$
\end{lemma}

In section 2, we saw that we could define path composition using path induction. While we would hope that, just like the previous lemma,
we could define morphism composition definitionally, that is not the case. In sHoTT,
the slogan is "some types are infinity categories" and this is the first place we see why that may be. Not all types come with the
structure of a category, but a single characterization allows us to talk about the types that do; the types that have categorical structure
are the ones that support composition. Our system is arranged in such a way that all categorical structure follows from composition.
For a programming languages analogy, composition of paths is a parametrically polymorphic 
function while composition of morphisms is a type class polymorphic function. That leads us to the following definition.
\begin{definition}[\cite{riehl_type_2017}, 5.3]
    \label{def:def3.3}
    Let $A$ be a type, $x,y,z : A$ and $f : \ho[A]{x}{y}$ and $g : \ho[A]{y}{z}$. We say $A$ is \textbf{Segal} if the type $${\sm{h : \ho[A]{x}{z}} \hot[A]{f}{g}{h}} $$
    is contractible. The unique inhabitant is denoted by $(g\cdot f, \comp{g}{f}).$
\end{definition}

In other words two composable morphisms specify a 2-morphism in a Segal type. To see this in action, we can show that a square, which
is the pushout product of two copies of 2-morphisms, is a proposition in a Segal type:
\begin{lemma}
    \todo{change this proof to say summing over the diagonal in a square is contractible in a segal type.}
    Given a Segal type $A$ and compatible morphisms $f : \ho[A]{a}{c}$, \linebreak $g:\ho[A]{b}{d}$, $p:\ho[A]{a}{b}$, and $q : \ho[A]{c}{d}$ then
    \begin{equation}
        \ndexten{\mathbbm{2}\times\mathbbm{2}}{A}{\partial \square}{[f,q,p,g]}
    \end{equation}
    where $\partial\square$ is the square boundary
\[
  \begin{tikzcd} 
        a \arrow[r, "f"] \arrow[d, "p"']   
        & c \arrow[d, "q"] 
         \\ 
        b \arrow[r, "g"']  & d
    \end{tikzcd} 
\]
    is contractible.
\end{lemma}
We can alternatively say that a horns
decides simplices in a Segal type. More formally,

\begin{lemma}[\cite{riehl_type_2017}, 5.5]
\label{lem:segalisrestrict}
A type $A$ is Segal if and only if the restriction map
$$
(\tsx \to A) \to (\tohorn \to A) 
$$
is an equivalence. 
\end{lemma}
In the proof of \ref{lem:segalisrestrict}, \cite{riehl_type_2017} proves along the way that there is an equivalence
\begin{equation}
    \sm{h : \ho[A]{x}{z}}\homtwo{A}(x,y,z,f,g,h) \simeq \ndexten{\tsx}{A}{\tohorn}{[x,y,z,f,g]}.
\end{equation}
As we will see, it is also useful to consider summing over the other sides of a 2-simplex.
\begin{corollary}
    In any type $A$, there are equivalences
    \begin{equation}
        \label{rem:sumoverfistt}
    \sm{f : \ho[A]{x}{y}}\homtwo{A}(x,y,z,f,g,h) \simeq \ndexten{\tsx}{A}{\tthorn}{[x,y,z,g,h]}
    \end{equation}
    and
    \begin{equation}
    \sm{g : \ho[A]{y}{z}}\homtwo{A}(x,y,z,f,g,h) \simeq \ndexten{\tsx}{A}{\tzhorn}{[x,y,z,f,h]}.
    \end{equation}
\end{corollary}
Given a segal type $A$, we have two ways of saying two morphisms $f,g : \ho[A]{x}{y}$ are the same. We can either:
\begin{enumerate}
    \item Provide a path $f = g$
    \item Provide a 2-simplex $\homtwo{A}(x,x,y,\idhom{x},f,g)$
\end{enumerate}
It turns out that both of these coincide.
\begin{lemma}
    \label{lem:pathis2mor}
    For any $f,g : \ho[A]{x}{y}$ in a Segal type $A$, the natural map
    $$f = g \to \homtwo{A}(x,x,y, \idhom{x}, f,g)$$
    is an equivalence.
\end{lemma}
More generally, composition and equality correlate.
\begin{lemma}
    \label{lem:compequalis2mor}
    For any $f: \ho[A]{x}{y}$, $g : \ho[A]{y}{z}$, and $h : \ho[A]{x}{z}$ in a Segal type $A$, the natural map
    $$g \cdot f = h \to \homtwo{A}(x,y,z,f,g,h)$$
    is an equivalence.
\end{lemma}
And more generally, commutative squares correspond to morphisms that commute.
\begin{lemma}
    \todo{look over this proof carefully}
    \label{lem:compissquare}
    Given a Segal type $A$ and compatible morphisms $f : \ho[A]{a}{c}$, \linebreak $g:\ho[A]{b}{d}$, $p:\ho[A]{a}{b}$, and $q : \ho[A]{c}{d}$ then
    \begin{equation}
        \ho[{(\lam{(x,y)}\ho[A]{x}{y})(\lam{i}(pi,qi))}]{f}{g} \simeq g \cdot p = q \cdot f
    \end{equation}
\end{lemma}
\begin{proof}
By lemma~\ref{lem:exten_curry},
\begin{equation*}
    \ho[{(\lam{(x,y)}\ho[A]{x}{y})(\lam{i}(pi,qi))}]{f}{g} \simeq \ndexten{\mathbbm{2}\times\mathbbm{2}}{A}{\partial \square}{[f,q,p,g]}
\end{equation*}
Since $\mathbbm{2}\times\mathbbm{2}$ is the pushout of two copies of $\tsx$ along the diagonal faces, we can push our equivalence to the type
\begin{equation*}
    \sm{h : \ho[A]{a}{d}}\left( \homtwo{A}(a,c,d,f,q,h) \times \homtwo{A}(a,b,d,p,g,h)\right)
\end{equation*}
By lemma~\ref{lem:sigisassoc}, this is equivalent to
\begin{equation*}
    \sm{(h, t) : \sm{h:\ho[A]{a}{d}}\homtwo{A}(a,c,d,f,q,h) } \homtwo{A}(a,b,d,p,g,h).
\end{equation*}
Lemma~\ref{lem:contractibleissimple} allows us to contract the base to give
\begin{equation}
    \homtwo{A}(a,b,d,p,g,q\cdot f),
\end{equation}
which by lemma~\ref{lem:compequalis2mor} is equivalent to
\begin{equation}
    g \cdot p = q \cdot f
\end{equation}
\end{proof}
\end{document}