\documentclass[main.tex]{subfiles}

\begin{document}
The astute reader will have noticed that having defined Segal types, we have two competing notions of equality. There is a tension
between isomorphisms between objects and paths between objects, which causes poor behavior. We are then interested in types where these
two notions coincide. This motivates the next type class definition. First, we must make precise what it means to be an isomorphism.
\begin{definition}
    For any Segal type $A$ and $f : \ho[A]{x}{y}$, we write
    \[
    \iso(f) \defeq \left ( \sm{g :\ho[A]{y}{x}}g \cdot f = \idhom{x} \right ) \times \left ( \sm{h : \ho[A]{y}{x}} f \cdot h = \idhom{y} \right )
    \]
    and say that $f$ is an \textbf{isomorphism} if $\iso(f)$ is inhabited. Similarly, for any $x,y : A$, we write \textbf{the type of isomorphisms}
    from $x$ to $y$ as 
    \[
    x \cong y \defeq \sm{f : \ho{A}{x}{y}}\iso(f)
    \]
\end{definition}
This coincides with our notion of a bi-invertible map. Just like in HoTT, we can show that inverses and bi-invertible maps are logically
equivalent.

\begin{lemma}[RS17 10.1]
    For any Segal type $A$ and morphism $f : \ho[A]{x}{y}$, $f$ is an isomorphism if and only if there is a $g : \ho[A]{y}{x}$ such that 
    $g \cdot f = \idhom{x}$ and $f \cdot g = \idhom{y}$.
\end{lemma}
We can also show that our type of isomorphisms is a proposition.
\begin{lemma}[RS17 10.2]
    For any Segal type $A$ and morphism $f : \ho[A]{x}{y}$, the type $\iso(f)$ is a proposition.
\end{lemma}
Now, we can start to reduce the tension between isomorphisms and identities. First, we can relate them by a map.
\begin{lemma}[RS17 10.5]
    \begin{equation}
        \idi : \prd{x,y:A}x =_A y \to x \cong_A y
    \end{equation}
    defined by path induction and the equation $\idi(\refl{x}) \defeq \idhom{x}$. If $A$ is a Segal type and 
    $\idi_{x,y}$ is an equivalence for all $x,y$, we say $A$ is \textbf{Rezk-complete} and that $A$ is a \textbf{Rezk type}.
\end{lemma}
\end{document}