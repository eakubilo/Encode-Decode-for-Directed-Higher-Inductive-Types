\documentclass[main.tex]{subfiles}

\begin{document}
\setlength{\perspective}{3pt}

\begin{figure}
    \begin{subfigure}[]{.48\textwidth}
        \[\begin{tikzcd}
	        {(0)} && {(1)}
	        \arrow[from=1-1, to=1-3]
        \end{tikzcd}\]
        \caption{$\mathbbm{2}$}
    \end{subfigure}
    \begin{subfigure}[]{.48\textwidth}
        \[\begin{tikzcd}
            {(0,1)} && {(1,1)} \\
            \\
            {(0,0)} && {(1,0)}
            \arrow[from=1-1, to=1-3]
            \arrow[from=3-1, to=1-1]
            \arrow[from=3-1, to=3-3]
            \arrow[from=3-3, to=1-3]
        \end{tikzcd}\]
        \caption{$\mathbbm{2}\times\mathbbm{2}$}
    \end{subfigure}
    \linebreak
    \begin{subfigure}[]{\textwidth}
        \centering
        \[\begin{tikzcd}[row sep={40,between origins}, column sep={40,between origins}]
      &[-\perspective] (0,1,1) \ar{rr} &[\perspective] &[-\perspective] (1,1,1) \\[-\perspective]
    (0,0,1) \ar{ur}\ar[crossing over]{rr}  & & (1,0,1) \ar{ur}  \\[\perspective]
      & (0,1,0)  \ar{uu}\ar{rr}  & &  (1,1,0)\ar{uu} \\[-\perspective]
    (0,0,0) \ar[uu]\ar{ur} \ar{rr} && (1,0,0) \ar{ur}\ar[uu, crossing over]
\end{tikzcd}\]
    \caption{$\mathbbm{2}\times\mathbbm{2}\times\mathbbm{2}$}
    \end{subfigure}
    \caption{First three powers of $\mathbbm{2}$}
    \label{fig:importantcubes}

\end{figure}
The foundation of sHoTT is a \textit{three-layer type theory with shapes}. More specifically, it is a type theory that consists 
of a \textbf{cube layer} at the top, which is a simple type theory with finite products. Cubes in sHoTT are finite powers of 
$\mathbbm{2}$, such as those in figure~\ref{fig:importantcubes}. The cube layer is formally the rules
\begin{mathpar}
    \tw~\cube \and
    0:\tw \and
    1:\tw
  \end{mathpar}


We have the \textbf{tope layer} sitting under the cube layer. The tope layer includes the binary relation $\leq$ that turns
$\mathbbm{2}$ into the strict partial order with bottom element $0$ and top element $1$ \cite{kudasov_formalizing_2023}. The tope layer allows us to talk 
about polytopes lying in cubes. That is, the tope layer is a two layer type theory that includes an extra, separate context for cubes. Judgements in the tope
 layer are generally expressed in the form
\begin{mathpar}
    \Xi\mid\Phi\vdash \phi
\end{mathpar}
\begin{figure}
    \begin{subfigure}[]{.48\textwidth}
        \[\begin{tikzcd}[execute at end picture={
            \foreach \Nombre in  {A,B,C}
              {\coordinate (\Nombre) at (\Nombre.center);}
            \fill[cof,opacity=0.3] 
              (A) -- (B) -- (C) -- cycle;
            }]
             && |[alias=C]|{(1,1)} \\
            \\
            |[alias=A]|{(0,0)} && |[alias=B]|{(1,0)}
            \arrow[from=3-1, to=1-3]
            \arrow[from=3-1, to=3-3]
            \arrow[from=3-3, to=1-3]
        \end{tikzcd}\]
        \caption{$\tsx \defeq \shape{\langle t,s \rangle }{\mathbbm{2} \times \mathbbm{2}}{s \leq t}$}
    \end{subfigure}
    \linebreak
    \begin{subfigure}[]{\linewidth}
        \[\begin{tikzcd}
             && |[alias=C]|{(1,1)} \\
            \\
            |[alias=A]|{(0,0)} && |[alias=B]|{(1,0)}
            \arrow[from=3-1, to=1-3]
            \arrow[from=3-1, to=3-3]
            \arrow[from=3-3, to=1-3]
        \end{tikzcd}\]
        \caption{$\pardt \defeq \shape{\langle t,s \rangle }{\mathbbm{2} \times \mathbbm{2}}{(0 \equiv s \leq t) \lor (s \equiv t) \lor (s \leq t \equiv 1)}$}
    \end{subfigure}
    \begin{subfigure}[]{\linewidth}
        \[\begin{tikzcd}
             (0,1)&& |[alias=C]|{(1,1)} \\
            \\
            |[alias=A]|{(0,0)} && |[alias=B]|{(1,0)}
            \arrow[from=3-1, to=1-1]
            \arrow[from=1-1, to=1-3]
            \arrow[from=3-1, to=3-3]
            \arrow[from=3-3, to=1-3]
        \end{tikzcd}\]
        \caption{$\pards \defeq \shape{\langle t,s \rangle }{\mathbbm{2} \times \mathbbm{2}}{( s \leq t \equiv 0) \lor (1 \equiv s \leq t) \lor (0 \equiv s \leq t) \lor (s \leq t \equiv 1)}$}
    \end{subfigure}
    \caption{Some important shapes}
    \label{fig:importantshapes}

\end{figure}

The tope layer also admits operations of finite disjunction, conjunction, equality, and inequality. The 
operations in the top layer allow us to talk about shapes as seen as in figure~\ref{fig:importantshapes}. The formal rules
of the cube layer are detailed in Appendix A.

%check here https://tex.stackexchange.com/questions/360083/filling-of-diagrams-using-tikzcd

\ifSubfilesClassLoaded{%
    \bibliographystyle{plain}
    \bibliography{citations.bib}
}{}
\end{document}