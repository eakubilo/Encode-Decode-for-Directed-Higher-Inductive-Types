\documentclass[main.tex]{subfiles}
\begin{document}
Rather than probing for categorical structure, we can instead probe for groupoid structure. Specifically, a type that has no
non-trivial simplicial data is a groupoid. This is captured in the following definition.
\begin{definition}[\cite{riehl_type_2017}, 7.1]
    For any type $A$, there is a map
    \begin{equation}
        \ida : \prd{x,y:A}x =_A y \to \ho[A]{x}{y}
    \end{equation}
    defined by path induction and the equation $\ida(\refl{x}) \defeq \idhom{x}$. We say $A$ is a \textbf{discrete type} if
    $\ida_{x,y}$ is an equivalence for all $x,y : A$.
\end{definition}

We can show that discrete types have the structure of a category. 
\begin{lemma}[\cite{riehl_type_2017}, 7.3]
    \label{lem:discretetypesseg}
    If $A$ is a discrete type them $A$ is Segal.
\end{lemma}
\end{document}