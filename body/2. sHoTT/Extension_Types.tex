\documentclass[main.tex]{subfiles}

\begin{document}
Extension types are important as they allow an easy interface to have types that depend on shapes and \textit{cofibrations}. To say more, suppose we have two shapes
 $\phi$ and $\psi$ and a judgement $ t : I \,| \,\phi \vdash \psi$. The judgement can intuitively be viewed as a shape inclusion, and we call this
  shape inclusion a cofibration. Moreover, given judgements $\psi \mid \Gamma \vdash A$, and $\phi \mid \Gamma \vdash a : A$, we write the type of extensions of $a$ as
\begin{equation}
    \label{eqn:extentype}
    \exten{t : I | \psi}{A}{\phi}{a}
\end{equation}
An extension type encodes the following diagram:
\[\begin{tikzcd}
	&& A \\
	\\
	\phi && \psi
	\arrow[two heads, from=1-3, to=3-3]
	\arrow[from=3-1, to=1-3]
	\arrow[hook, from=3-1, to=3-3]
	\arrow[bend right = 30, from=3-3, to=1-3]
\end{tikzcd}\]
\begin{remark}
For readability purposes, we note that the full formal specification of extension types can be found at \cref{fig:exten} and we adopt the following conventions:
\begin{enumerate}
\item When $A$ does not depend on a shape, instead of $\exten{t : I | \psi}{A}{\phi}{a}$ we write $\ndexten{\shape{t}{I}{\psi}}{A}{\phi}{a}$.
\item When $\phi$ is $\bot$, we observe that $a$ is $\recbot$, so instead of $$\exten{\preshape{t}{I}{\psi}}{A}{\bot}{\recbot},$$ we write $$\prd{\preshape{t}{I}{\psi}}A$$ and instead of $$\ndexten{\shape{t}{I}{\psi}}{A}{\bot}{\recbot},$$ we write $$\shape{t}{I}{\psi} \to A.$$
\item Since we have extension types defined, we can think of $A$ literally as a function $$A : \shape{t}{I}{\psi} \to \U$$ with a section $a : \prd{\preshape{t}{I}{\phi}}A(t)$ and write its extension type as\linebreak
$$\exten{\preshape{t}{I}{\psi}}{A(t)}{\phi}{a}.$$
\end{enumerate}
\end{remark}

The syntax of an extension type is not just suggestive. They are intended to be a generalization of dependent functions and as a
 result we can prove similar facts about extension types as we can about functions. To start, we know that the arguments of a 
 function commute. That is, 
\[
(X \to Y) \to Z \simeq Y \to (X \to Z)
\]
and
\[
\prd{x : X}\prd{y : Y}Z(x,y) \simeq \prd{y : Y}{x : X}Z(x,y).
\]
The commutativity of the arugments of an extension type also follow.
\begin{lemma}[\cite{riehl_type_2017}, 4.1]
    If $t : I \, \mid \, \phi \vdash \psi$ and $X : \U$, while $Y : \{t : I \, \mid \phi\} \to X \to \U$ and $f : \prd{t : I \, \mid\, \phi }{x : X}Y(t,x)$ then
    \[
    \exten{t : I \, \mid \, \psi }{\prd{x : X}Y(t,x)}{\phi}{f} \simeq \prd{x : X}\exten{t : I \, \mid \, \psi}{Y(t,x)}{\phi}{f}
    \]
\end{lemma}
Similarly, we can curry the arguments of a function. That is, there is an equivalence $$X \to Y \to Z \simeq (X \times Y) \to Z.$$
Extension types also behave similarly, allowing for currying and swapping in multiple argument functions.
\begin{lemma}[\cite{riehl_type_2017}, 4.2]
    \label{lem:exten_curry}
    If $t : I \mid \phi \vdash \psi$ and $s : J \mid \chi \vdash \zeta$, while
    \begin{equation*}
    X : \{ t : I \mid \psi\} \to \{ s : J \mid \zeta \} \to \U
    \end{equation*}
    and $f:\prd{\langle t,s \rangle : I \times J \mid (\phi \land \zeta) \lor (\psi \land \chi)}X(t,s),$ then
    \begin{align*}
        &\exten{t:I\mid\psi}{\exten{s : J \mid \zeta}{X(t,s)}{\chi}{\lam{s}f\langle t,s \rangle}}{\phi}{\lam{t}f\langle t,s \rangle} \\
        &\qquad\qquad\simeq \exten{\langle t,s \rangle : I \times J \mid \psi \land \zeta}{X(t,s)}{(\phi \land \zeta) \lor (\psi \land \chi)}{f} \\
        &\qquad\qquad\simeq \exten{s : J \mid \zeta}{\exten{t : I \mid \psi}{X(t,s)}{\phi}{\lam{s}f\langle t,s \rangle}}{\chi}{\lam{s}{t}f\langle t s \rangle}
    \end{align*}
\end{lemma}

We can also extend the type theoretic axiom of choice, which states that there is an equivalence 
\[
\prd{x:X}\sm{y:Y(x)}Z(x,y) \simeq \sm{f:\prd{x:X}Y(x)}\prd{x:X}Z(x,f(x))
\]
to extension types:
\begin{lemma}[\cite{riehl_type_2017}, 4.3]
    \label{lem:taocwet}
    If $\preshape{t}{I}{\phi \vdash \psi}$, while $X : \shape{t}{I}{\psi} \to \U$ and $Y : \prd{\preshape{t}{I}{\psi}}(X(t) \to \U)$, while
    $a : \prd{\preshape{t}{I}{\phi}}X(t)$ and $b : \prd{\preshape{t}{I}{\phi}}Y(t,a(t))$, then 
    \begin{align*}
        &\exten{\preshape{t}{I}{\psi}}{\sm{x : X(t)} Y(t,x)}{\phi}{\lam{t}(a(t),b(t))}  \\
        &\simeq\sm{f : \exten{\preshape{t}{I}{\psi}}{X(t)}{\phi}{a}}\exten{\preshape{t}{I}{\psi}}{Y(t,f(t))}{\phi}{b}.
    \end{align*}
\end{lemma}
While the following two lemmas do not have direct analogus with the usual function type, they are nonetheless extremely useful:
\begin{lemma}[\cite{riehl_type_2017}, 4.4]
    \label{lem:compofexten}
    Suppose $\preshape{t}{I}{\phi \vdash \psi}$ and $\preshape{t}{I}{\psi \vdash \chi}$, and that $X : \shape{t}{I}{\chi} \to \U$ and $a : \prd{\preshape{t}{I}{\phi}}X(t)$. Then,
    $$
        \exten{\preshape{t}{I}{\chi}}{X}{\phi}{a} \simeq  \sm{f : \exten{\preshape{t}{I}{\psi}}{X}{\phi}{a}}\exten{\preshape{t}{I}{\chi}}{X}{\psi}{f}
    $$
\end{lemma}
\begin{lemma}[\cite{riehl_type_2017}, 4.5]
    \label{lem:disjunctexten}
    Suppose $t:I \vdash \phi$ and $t : I \vdash \psi$ and that $X : \shape{t}{I}{\phi \lor \psi} \to \U$ and $a : \prd{\preshape{t}{I}{\psi}}X(t)$. Then,
    $$
        \exten{\preshape{t}{I}{\phi \lor \psi}}{X}{\psi}{a} \simeq \exten{\preshape{t}{I}{\phi}}{X}{\phi \land \psi}{\lam{t}a(t)}
    $$
\end{lemma}
In HoTT, the relative function extensionality axiom allows us to conclude that $\prd{x:X}B(x)$ is contractible if each $B(x)$ is contractible.
We can state a similar axiom using extension types.
\begin{axiom}[\cite{riehl_type_2017}, relative function extensionality, 4.6]
\label{axi:relfunext}
Supposing $\preshape{t}{I}{\phi \vdash \psi }$ and that $A : \shape{t}{I}{\phi} \to \U$ is such that each $A(t)$ is contractible, and
moreover $a : \prd{\preshape{t}{I}{\phi}}A(t)$, then $\exten{\preshape{t}{I}{\psi}}{A(t)}{\phi}{a}$ is contractible.
\end{axiom}
We can also show that function extensionality follows from axiom~\cref{axi:relfunext}.
\begin{lemma}[\cite{riehl_type_2017}, 4.8]
    \label{lem:simpfunext}
    Assuming axiom~\cref{axi:relfunext}, given a family $A : \shape{t}{I}{\psi}$ and $a : \prd{\preshape{t}{I}{\phi}}A(t)$, and $f,g : \exten{\preshape{t}{I}{\psi}}{A(t)}{\phi}{a}$, 
    the map
    \[
    (f=g)\to \exten{\preshape{t}{I}{\psi}}{f(t)=g(t)}{\phi}{\lam{t}\refl{}}
    \]
    is an equivalence.
\end{lemma}
As is done in \cite{riehl_type_2017}, we assume axiom~\cref{axi:relfunext}.
\ifSubfilesClassLoaded{%
    \bibliographystyle{plain}
    \bibliography{citations.bib}
}{}
\end{document}