\documentclass[main.tex]{subfiles}

\begin{document}
The connection between type theory and topology is evident in the concept of paths. Paths in type theory, analogous to topological paths, act as morphisms in higher groupoids. The constant path at $x$ is $\refl{x}$. Using path induction, we can show that path equality forms an equivalence relation, mirroring the behavior of topological paths. The following lemmas demonstrate this property.

\begin{lemma}[\cite{program_homotopy_2013}, 2.1.1]
    For any $x,y : A$ there is a function
    $$x = y \to y = x$$
    denoted by $p \mapsto p^{-1}$, such that $\refl{x}^{-1} \equiv \refl{x}$. We call $p^{-1}$ the \textbf{inverse} of $p$.
\end{lemma}
We can also show that path types are transitive.
\begin{lemma}[\cite{program_homotopy_2013}, 2.1.2]
    For any $x,y,z : A$ there is a function
    $$x = y \to y = z \to x = z$$
    denoted by $p \mapsto q \mapsto q \cdot p$. We call $q \cdot p$ the \textbf{concatenation} of $p$ and $q$.
\end{lemma}
However, in the proof-relevant setting of type theory, simply having functions demonstrating these equivalence relation properties is not enough to fully capture the higher groupoid structure. We need to ensure that these operations behave consistently with each other. This is where the concept of coherence laws comes in. These laws assert the equality of different ways of manipulating paths, reflecting the fact that proofs themselves can be considered equal.

\begin{lemma}[\cite{program_homotopy_2013}, 2.1.4]
    For any type $A$, $x,y,z,w : A$ $p : x =y$, $q : y = z$, and $r : z = w$ we have the following:
    \begin{enumerate}
        \item $p = p \cdot \refl{y}$ and $p = \refl{x} \cdot p$.
        \item $p^{-1} \cdot p = \refl{y}$ and $p \cdot p^{-1} = \refl{x}$.
        \item $(p^{-1})^{-1} = p$.
        \item $p \cdot (q \cdot r) = (p \cdot q) \cdot r$.
    \end{enumerate}
\end{lemma}
The lemmas presented so far illustrate the one-dimensional structure of paths. To fully realize the vision of types as higher groupoids, we must extend these concepts and demonstrate analogous properties for higher-dimensional paths – paths between paths, paths between paths between paths, and so on.
\end{document}