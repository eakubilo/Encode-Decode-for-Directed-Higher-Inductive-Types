\documentclass[main.tex]{subfiles}

\begin{document}
Paths in type theory mirror paths in topology, which are morphisms in higher groupoids. The constant path
 $c_x$ is represented by $\refl{x}$. Path induction allows us to probe the more complex structure. For example, 
 we can show that paths are an equivalence relation, just like paths in topology. The following lemma shows the path type
  is symmetric.
\begin{lemma}[HoTT Book 2.1.1]
    For any $x,y : A$ there is a function
    $$x = y \to y = x$$
    denoted by $p \mapsto p^{-1}$, such that $\refl{x}^{-1} \equiv \refl{x}$. We call $p^{-1}$ the \textbf{inverse} of $p$.
\end{lemma}
We can also show that path types are transitive.
\begin{lemma}[HoTT Book 2.1.2]
    For any $x,y,z : A$ there is a function
    $$x = y \to y = z \to x = z$$
    denoted by $p \mapsto q \mapsto q \cdot p$. We call $q \cdot p$ the \textbf{concatenation} of $p$ and $q$.
\end{lemma}
Due to the setting we are in, it is not enough to define maps that show paths are an equivalence relation if we want to say
 types are higher groupods. We must show that they satisfy coherence laws as well, considering we are in a proof-relevent setting.
\begin{lemma}[HoTT Book 2.1.4]
    For any type $A$, $x,y,z,w : A$ $p : x =y$, $q : y = z$, and $r : z = w$ we have the following:
    \begin{enumerate}
        \item $p = p \cdot \refl{y}$ and $p = \refl{x} \cdot p$.
        \item $p^{-1} \cdot p = \refl{y}$ and $p \cdot p^{-1} = \refl{x}$.
        \item $(p^{-1})^{-1} = p$.
        \item $p \cdot (q \cdot r) = (p \cdot q) \cdot r$.
    \end{enumerate}
\end{lemma}
We are still not done, though. To truly show that types satisfy higher groupoid structure, we need to extend the lemmas we just stated
to higher dimensions of paths.
\end{document}