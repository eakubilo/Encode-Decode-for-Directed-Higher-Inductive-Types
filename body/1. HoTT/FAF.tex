\documentclass[main.tex]{subfiles}

\begin{document}
A crucial aspect of Homotopy Type Theory is the principle that functions between types act consistently with the path structure, mirroring the behavior of functors in category theory. This means that functions not only map terms to terms but also transform paths between terms in a coherent way. To formalize this, we first establish that functions "respect equality," meaning they map equal terms to equal terms.
\begin{lemma}[\cite{program_homotopy_2013}, 2.2.1]
    For any function $f : A \to B$ and $x,y : A$, there is a function
    $$\apfunc{f}: x = y \to f(x) = f(y)$$
    such that each $x : A$, we have $\apfunc{f}(\refl{x}) \equiv \refl{f(x)}$.
\end{lemma}

Building on this fundamental property, we can then demonstrate the full functorial behavior of functions on paths.

\begin{lemma}[\cite{program_homotopy_2013}, 2.2.2]
    For functions $f : A \to B$ and $g : B \to C$ and paths $p : x = y$ and $q : y = q$, we have:
    \begin{enumerate}
        \item $\apfunc{f}(p \cdot q) = \apfunc{f}(p)\cdot \apfunc{f}(q)$.
        \item $\apfunc{f}(p^{-1}) = (\apfunc{f}(p))^{-1}$.
        \item $\apfunc{g}(\apfunc{f}(p)) = \apfunc{g \circ f}(p)$.
        \item $\apfunc{\id{A}}(p) = p$.
    \end{enumerate}
\end{lemma}
\end{document}