\documentclass[main.tex]{subfiles}

\begin{document}
In our groupoid interpretation of type theory, we aim to show that it is reasonable to think of type families as "fibrations".
That is, that they satisfy some suitable lifting property. Before we can properly state the lifting property, we give the "indiscernability of identity" principle 
a new name for our setting.
\begin{lemma}[Transport, HoTT Book 2.3.1]
    For any type family $P : A \to \U$ and path $p : x =_A y$, there is a function 
    \[\text{transport}^P(p, -) : P(x) \to P(y),\]
    or similarly denoted as $p_* : P(x) \to P(y)$ when clear from context.
\end{lemma}
Thus, transporting is our equivalent of path lifting. When viewing a type family, such as $P : A \to \U$ as a fibration, we do so as
as a fibration $\sm{x : a}P(x) \to A$. Thus, given a potential source $u : P(x)$, transporting allows us to lift a path $p : x \to y$
to a path from $u$ to $p_*(u)$. This is stated concretely in the following lemma.
\begin{lemma}[Path lifting property, HoTT Book 2.3.2]
    Let $P : A \to \U$ be a type family and let $u : P(x)$ for some $x : A$. Then, for any $p : x = y$, there is a path
    $$\text{lift}(u,p) : (x,u) = (y, p_*(u)).$$
    in $\sm{x : A}P(x)$ such that $\fst(\text{lift}(u,p)) = p$.
\end{lemma}
\end{document}