\documentclass[main.tex]{subfiles}

\begin{document}
In the realm of type theory, the familiar notion of a set can be understood through a lens that emphasizes the identity of its elements. Specifically, a set in type theory is a type that is distinguished by its objects, possessing no additional ``higher homotopical data". This leads to the following definition:
\begin{definition}[\cite{program_homotopy_2013}, 3.1.1]
    A type $A$ is said to be a \textbf{set} if the type 
    \[
    \mathsf{isSet(A)} \defeq \prd{x,y : A}\prd{p,q : x=_A y} p =_{x=_A y} q
    \]
\end{definition}

Building upon this foundation, type theory organizes types into a hierarchy based on their level of truncation, which indicates the highest dimension at which non-trivial homotopical information exists. Sets occupy the level of $0$-types. Of particular interest are the types that reside ``below" sets in this hierarchy: contractible types ($-2$ types), which contain precisely one object, and propositions ($-1$ types), which contain at most one object. We now define these fundamental types, which are essential for understanding the broader concept of $n$-types.

\begin{definition}[\cite{program_homotopy_2013}, 3.11.1]
    A type $A$ is said to be \textbf{contractible} if the type
    \[
    \mathsf{isContractible(A)} \defeq \sm{x : A}\prd{y : A} x =_A y
    \]
\end{definition}

\begin{definition}[\cite{program_homotopy_2013}, 3.3.1]
    A type $A$ is said to be a \textbf{proposition} if the type
    \[
    \mathsf{isProp(A)} \defeq \prd{x,y : A} x=_A y
    \]
\end{definition}

HoTT also allows us tools to approximate types up to the propositional level, allowing for traditional concepts like the set theoretic subset to have analogues in type theory.
\begin{definition}{\cite{program_homotopy_2013}, 3.7}
    For any type $A$, there is another type $\trunc{}{A}$ called the \textbf{propositional truncation of} $A$, defined by the constructors
    \begin{itemize}
        \item For all $a : A$ we have $|a| : \trunc{}{A}$.
        \item For all $x,y : \trunc{}{A}$ we have $x =_{\trunc{}{A}} y$
    \end{itemize} 
\end{definition} 

Apart from approximating types, we can extend our the definition of propositions and contractible types to higher dimensions. In a proposition, its identity types are contractible. We expect a set to be a type that is distinguished by its objects, thus it must have no non-trivial paths. Thus, a set is a type whose identity types are propositions. This pattern generalizes to all higher dimensions as seen in the following definition: 

\begin{definition}[\cite{program_homotopy_2013}, 7.1.1]
    For any type $A$, we say it is a $-2$ type if it is contractible and an $n+1$ type if its identity types are $n$ types.
\end{definition}

We also have a fundamental result that relates the truncation level of a type, the highest level of which there is non-trivial homotopical data, to its identity types.

\begin{lemma}[\cite{rijke2022introductionhomotopytypetheory}, 12.4.4]
    \label{lem:ntypeidenntype}
    For any type $A$, if it is an $n$-type, then its identity types are $n$-types.
\end{lemma}

A fundamental result in homotopy type theory is the contractibility of based path spaces. This result, often referred to as \textbf{singleton contractibility}, is formally stated as follows:

\begin{lemma}[\cite{program_homotopy_2013}, 3.11.8]
    \label{lem:contractiblesingletons}
    For any type $A$ and $a : A$, the type $\sm{x:A}a=x$ is contractible.
\end{lemma}
When working with Sigma types (dependent sum types), the property of contractibility in either the base type or the fiber types leads to significant simplifications in the type structure, up to type equivalence. This is a crucial tool for simplifying type expressions.
\begin{lemma}[\cite{program_homotopy_2013}, 3.11.9]
    \label{lem:contractibleissimple}
    For any type family $P : A \to \U$:
    \begin{enumerate}
        \item If each $P(x)$ is contractible, then $\sm{x:A}P(x)\simeq A$.
        \item If $A$ is contractible with center of contraction $a:A$, then $\sm{x:A}P(x) \simeq P(a)$.
    \end{enumerate}
\end{lemma}
Finally, it is important to note that logical principles arise naturally within type theory, and are not solely confined to the definition of $n$-types. We present a principle that is logically weaker than the full axiom of choice from classical set theory, known as \textbf{the type theoretic axiom of choice}.
\begin{lemma}[\cite{program_homotopy_2013}, 1.7]
    \label{lem:ttaoc}
    For a type $X$ and type families $A : X \to \U$ and $P : \prd{x:X}A(x) \to \U$,
    \begin{equation}
        \left(\prd{x : X}\sm{a : A(x)}P(x,a)\right) \simeq \left(\sm{g : \prd{x:X}A(x)}\prd{x:X}P(x, g(x))\right)
    \end{equation}
\end{lemma}





\end{document}