\documentclass[main.tex]{subfiles}

\begin{document}
In type theory, we can recover the concept of a set by asking for a type that is determined by its objects. That is, a set is 
a type with no higher homotopical data.
\begin{definition}
    A type $A$ is said to be a \textbf{set} if the type 
    \[
    \mathsf{isSet(A)} \defeq \prd{x,y : A}\prd{p,q : x=_A y} p = q
    \]
\end{definition}

There are two important classes of sets that we often overlook when doing classical math: those that contain \textit{exactly one} object,
and those that contain \textit{at most one} object. These two classes form what we call $-2$ types and $-1$ types, respectively, with a set
forming a $0$ type. We can actually form a hierarchy of classes of types following this interfact, but for this thesis we only additionally
define the first two. 

\begin{definition}
    A type $A$ is said to be \textbf{contractible} if the type
    \[
    \mathsf{isContractible(A)} \defeq \sm{x : A}\prd{y : A} x = y
    \]
\end{definition}
So, contractible types correspond to $-2$ types.
\begin{definition}
    A type $A$ is said to be a \textbf{proposition} if the type
    \[
    \mathsf{isProp(A)} \defeq \prd{x,y : A} x= y
    \]
\end{definition}

Generally, based path spaces are contractible. We call this result \textbf{singleton contractibility}
\begin{lemma}
    \label{lem:contractiblesingletons}
    For any type $A$ and $a : A$, the type $\sm{x:A}a=x$ is contractible.
\end{lemma}
When dealing with Sigma types, the base or fibers being contractible allow us to greatly simplify the type presentations, up to equivalence.
\begin{lemma}
    \label{lem:contractibleissimple}
    For any type family $P : A \to \U$:
    \begin{enumerate}
        \item If each $P(x)$ is contractible, then $\sm{x:A}P(x)\simeq A$.
        \item If $A$ is contractible with center of contraction $a:A$, then $\sm{x:A}P(x) \simeq P(a)$.
    \end{enumerate}
\end{lemma}
Finally, logic does not just appear when defining $n$-types. We can state a principle that is derived from the axiom of choice
but slightly weaker, donned \textbf{the type theoretic axiom of choice}.
\begin{lemma}
    \label{lem:ttaoc}
    For a type $X$ and type families $A : X \to \U$ and $P : \prd{x:X}A(x) \to \U$,
    \begin{equation}
        \left(\prd{x : X}\sm{a : A(x)}P(x,a)\right) \simeq \left(\sm{g : \prd{x:X}A(x)}\prd{x:X}P(x, g(x))\right)
    \end{equation}
\end{lemma}





\end{document}