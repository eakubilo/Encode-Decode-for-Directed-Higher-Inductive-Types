\documentclass[main.tex]{subfiles}

\begin{document}
Beyond paths between terms, functions themselves can be related in type theory. A natural way to compare functions is through pointwise identification.
\begin{definition}[\cite{program_homotopy_2013}, 2.4.1]
    Let $P : A \to \U$ be a type family and $f,g : \prd{x : A}P(x)$. We write
    $f\sim g \defeq \prd{x : A} f(x) = g(x) $
    and say an inhabitant of $f \sim g$ is a \textbf{homotopy} from $f$ to $g$.
\end{definition}
Since homotopies are essentially pointwise paths, we expect them to share fundamental properties with path equality.

\begin{lemma}[\cite{program_homotopy_2013}, 2.4.2]
    Homotopy is an equivalence relation on the dependent function type. That is, for any type family $P : A \to \U$, there are maps
    \begin{align*}
        \prd{f : \prd{x : A}P(x)} f \sim f\\
        \prd{f,g : \prd{x:A}P(x)} \left ( f \sim g \right ) \to \left ( g \sim f \right )\\
        \prd{f,g,h : \prd{x : A} P(x)} \left ( f \sim g \right ) \to \left ( g \sim h \right ) \to \left (f \sim h\right )
    \end{align*}
\end{lemma}
Moving beyond functions, we can consider how to express that two types, $A$ and $B$, are "the same". In Homotopy Type Theory, simple path equality between types is insufficient to capture this notion fully. Drawing inspiration from category theory, the concept of \textbf{equivalence} provides a more robust notion of sameness.

A first attempt at defining equivalence might involve \textbf{quasi-inverses}. For a map $f : A \to B$, a quasi-inverse $g : B \to A$ would satisfy $f\circ g \sim \id{B}$ and $g \circ f \sim \id{A}$. The type of all quasi-inverses of $f$ can be written as:
\[
    \qinv(f) \defeq \sm{g : B \to A}((f\circ g \sim \id{B})\times(g \circ f \sim \id{A})).
\]    
While this notion aligns with the idea of equivalence in standard category theory, HoTT typically employs a different, more refined definition.

Instead of relying on a single quasi-inverse, we can decompose the notion of an inverse into two parts: left and right inverses. A function $g$ is a \textbf{left inverse} of $f: A \to B$ if $g \circ f \sim \id{A}$, and the type of all left inverses is:
\begin{equation*}
    \linv(f) \defeq \sm{g : B \to A}g \circ f \sim \id{A}.
\end{equation*}
Similarly, $g$ is a \textbf{right inverse} of $f$ if $f \circ g \sim \id{B}$, and the type of all right inverses is:
\begin{equation*}
    \rinv(f) \defeq \sm{g : B \to A} f \circ g \sim \id{B}.
\end{equation*}
A map $f$ is then considered an \textbf{equivalence} (or \textbf{bi-invertible}) if it possesses both a left and a right inverse. The type of proofs that $f$ is an equivalence is defined as:
\begin{equation*}
    \iseq(f) \defeq \biinv(f) \defeq \linv(f) \times \rinv(f).
\end{equation*}


This notion of equivalence generalizes the familiar idea of an isomorphism or bijection between sets. If $A$ and $B$ are sets, an equivalence $f: A \to B$ corresponds precisely to a function that is both surjective and injective. We can recover these concepts within type theory.

\begin{definition}[\cite{program_homotopy_2013}, 4.6.1]
    For any $f : A \to B$
    \begin{enumerate}
        \item If for every $b:B$ the type $\trunc{}{\hfib{f}{b}}$ is inhabited, $f$ is said to be \textbf{surjective}.
        \item If for every $x,y : A$ the function $\apfunc{f}:(x=_A y) \to (f(x) =_B f(y))$ is an equivalence, $f$ is said to be an
            \textbf{embedding}.
    \end{enumerate}
\end{definition}
The connection between equivalences, surjectivity, and embeddings is fundamental:
\begin{lemma}[\cite{program_homotopy_2013}, 4.6.3]
    \label{lem:equivissurjemb}
    A function $f : A \to B$ is an equivalence if and only if it is both surjective and an embedding.
\end{lemma}
When working with type families, $P,Q : A \to \U$, we introduce \textbf{fiberwise maps} to relate them. A fiberwise map $f : \prd{x:A}P(x) \to Q(x)$ provides a way to map elements within each fiber. Such maps naturally induce maps between the total spaces of the type families.
\begin{definition}[\cite{program_homotopy_2013}, 4.7.5]
    For any type families $P,Q : A \to \U$ and fiberwise map $f : \prd{x:A}P(x) \to Q(x)$, we define
    \begin{equation}
        \tot(f) \defeq \lam{(x,u)}(x, f(x,u)) : \sm{x : A}P(x) \to \sm{x:A}Q(x)
    \end{equation}
\end{definition}

We can further define a \textbf{fiberwise equivalence}. A fiberwise map $f : \prd{x:A}P(x) \to Q(x)$ is a fiberwise equivalence if, for each $x:A$, the map $f(x) : P(x) \to Q(x)$ is an equivalence. This property is closely tied to whether the induced map on total spaces is an equivalence.

\begin{lemma}[\cite{program_homotopy_2013}, 4.7.7]
    \label{lem:equivistotequiv}
    For any type families $P,Q : A \to \U$ and fiberwise map $f : \prd{x:A}P(x)\to Q(x)$, $f$ is a fiberwise equivalence if and only if
    $\tot(f)$ is an equivalence.
\end{lemma}
Equivalences play a crucial role in transporting structure. If $e : A \to B$ is an equivalence, it allows us to relate total spaces over $A$ and $B$.
\begin{lemma}[\cite{program_homotopy_2013}, Exercise 2.17]
    \label{lem:equivbaseequivtot}
    For any type family $C : A \to \U$, types $A$ and $B$, and equivalence $e : A \to B$ with quasi-inverse $e^{-1}$, 
    \begin{equation}
        \sm{a : A}C(a) \simeq \sm{b:B}C(e^{-1}(b))
    \end{equation}
\end{lemma}

Finally, even path composition itself exhibits the structure of an equivalence.
\begin{lemma}[\cite{program_homotopy_2013}, Exercise 2.6]
    \label{lem:compisequiv}
    For any $p : x = y$, the function
    \begin{equation}
        (p \sq -) : (y = z) \to (x = z)
    \end{equation}
    is an equivalence.
\end{lemma}

\end{document}