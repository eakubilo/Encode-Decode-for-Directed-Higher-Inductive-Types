\documentclass[main.tex]{subfiles}

\begin{document}
Between functions, paths are not the only way to relate them. There is another natural relation between them that comes from
pointwise identification.
\begin{definition}
    Let $P : A \to \U$ be a type family and $f,g : \prd{x : A}P(x)$. We write
    $f\sim g \defeq \prd{x : A} f(x) = g(x) $
    and say an inhabitant of $f \sim g$ is a \textbf{homotopy} from $f$ to $g$.
\end{definition}
Since homotopies are really just pointwise paths, we expect them to satisfy the same properties as types. 
\begin{lemma}[HoTT Book 2.4.2]
    Homotopy is an equivalence relation on the dependent function type. That is, for any type family $P : A \to \U$, there are maps
    \begin{align*}
        \prd{f : \prd{x : A}P(x)} f \sim f\\
        \prd{f,g : \prd{x:A}P(x)} \left ( f \sim g \right ) \to \left ( g \sim f \right )\\
        \prd{f,g,h : \prd{x : A} P(x)} \left ( f \sim g \right ) \to \left ( g \sim h \right ) \to \left (f \sim h\right )
    \end{align*}
\end{lemma}
In HoTT, for a map $f : A \to B$, the notion of an inverse, which we denote its \textbf{quasi-inverse} and represent the type of all such
quasi-inverses with the type
\begin{equation}
    \qinv(f) \defeq \sm{g : B \to A}((f\circ g \sim \id{B})\times(g \circ f \sim \id{A})).
\end{equation} 
In category theory, this is a nice enough notion to represent "equivalence of categories" In HoTT, this type is not nice enough to 
to be the premiere notion of equivalence. Instead, we can split the inverse into two parts. The first we denote
as a \textbf{left inverse} and write the type of all left inverses as 
\begin{equation}
    \linv(f) \defeq \sm{g : B \to A}g \circ f \sim \id{A}.
\end{equation}
The second is denoted as a \textbf{right inverse} and the type of all right inverses is written as
\begin{equation}
    \rinv(f) \defeq \sm{g : B \to A} f \circ g \sim \id{B}.
\end{equation}
A map $f$ is a \textbf{bi-invertible} map if it has both a left and right inverse. That is, if the type
\begin{equation}
\biinv(f) \defeq \linv(f) \times \rinv(f)
\end{equation}
is inhabited. Bi-invertible maps are the premiere notion of equivalence for this text. That is,
\begin{equation}
    \iseq \defeq \biinv
\end{equation}

Given two type families $P,Q : A \to \U$, we say a map $f : \prd{x:A}P(x) \to Q(x)$ is a \textbf{fiberwise map}. Fiberwise maps induce maps
between total spaces.
\begin{definition}[HoTT Book 4.7.5]
    For any type families $P,Q : A \to \U$ and fiberwise map $f : \prd{x:A}P(x) \to Q(x)$, 
    \begin{equation}
        \tot(f) \defeq \lam{(x,u)}(x, f(x,u)) : \sm{x : A}P(x) \to \sm{x:A}Q(x)
    \end{equation}
\end{definition}

We can prove closure properties as well. We say that $f$ is a fiberwise equivalence if, for each $f(x) : P(x) \to Q(X)$ is an equivalence.

\begin{lemma}[HoTT Book 4.7.7]
    \label{lem:equivistotequiv}
    For any type families $P,Q : A \to \U$ and fiberwise map $f : \prd{x:A}P(x)\to Q(x)$, $f$ is a fiberwise equivalence if and only if
    $\tot(f)$ is an equivalence.
\end{lemma}
We can "transport" structure along equivalences as well to change base types in a total space.
\begin{lemma}[HoTT Book Exercise 2.17]
    \label{lem:equivbaseequivtot}
    If $e : A \to B$ is an equivalence with quasi-inverse $e^{-1}$, 
    \begin{equation}
        \sm{a : A}B(a) \simeq \sm{b:B}B(e^{-1}(b))
    \end{equation}
\end{lemma}

Path composition is also an equivalence.
\begin{lemma}[HoTT Book Exercise 2.6]
    \label{lem:compisequiv}
    For any $p : x = y$, the function
    \begin{equation}
        (p \sq -) : (y = z) \to (x = z)
    \end{equation}
    is an equivalence.
\end{lemma}

\end{document}