\documentclass[main.tex]{subfiles}

\begin{document}
When dealing with $\Sigma$ types, there are many techniques that we consider ubiquitous in our study. The first is that in the 
non-dependent case, we can show that products are symmetric:
\begin{lemma}
    \label{lem:sigissymm}
    For any two types $A$ and $B$, we have $A \times B \simeq B \times A$.
\end{lemma}
We can also show that associativity holds in all cases, even dependently:
\begin{lemma}[\cite{program_homotopy_2013}, Exercise 2.10]
    \label{lem:sigisassoc}
    For any type $A : \U$ and type families $B : A \to \U$ and $C : (\sm{x : A}B(x)) \to \U$, the following holds
    \begin{equation}
        \left( \sm{x : A}\sm{y : B(x)}C((x,y)) \right) \simeq \left( \sm{p : \sm{x:A}B(x)}C(p) \right)
    \end{equation}
\end{lemma}
\end{document}