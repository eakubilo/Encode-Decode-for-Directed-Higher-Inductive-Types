\documentclass[main.tex]{subfiles}

\begin{document}
As a warmup to more complicated directed higher inductive types, we study the various possible conceptions of the natural numbers in
 sHoTT. We start with the most familiar:
\begin{definition}
    Let $\N$ be the type generated by the constructors
    \begin{itemize}
       \item $0 : \N$
       \item For all $n : \N$, a term $\su(n) : \N$.
    \end{itemize}
    The induction principle for $\N$ states that for any type family $C : \N \to \U$ along with
    \begin{itemize}
       \item A point $c_0 : C(0)$
       \item A function $c_s : \prd{n:\N}C(n) \to C(\su(n))$
    \end{itemize}
   There is a function $f : \prd{n:\N}C(n)$ such that $f(0) \equiv c_0$ and for all $n : \N$, $f(\su(n)) \equiv c_s(n, f(n))$.
   
\end{definition}

While this incarnation of the natural numbers seems syntactically permissible, its behavior is poor. For example, we can try to 
characterize the morphism type between two arbitrary natural numbers. To do so, we define a map $\co : \N \to \N \to \U$ with the 
defining equations 
\begin{align*}
    &\co\,0\,0\defeq \Un&\\
    &\co\,0\,S(n) \defeq \vo&\\
    &\co\,S(n)\,0\defeq \vo&\\
    &\co\,S(n)\,S(m)\defeq \co \,n\,m&
\end{align*}
We immediately run into issues while trying to define $$\eco : \prd{n : \N}{m : \N} \ho[\N]{n}{m} \to \co\,n\,m.$$ To do so, we'd have to supply
a term for $\eco\,0\,S(n)$ and $\eco\,S(n)\,0$. Evaluating the type of such a term, we find that we are required to come up with a map
that takes a morphism of type $\ho[\N]{0}{S(n)}$ to a term of type $\co\,0\,S(n)\equiv \vo$. Since we cannot inhabit $\vo$, we need to show 
that $\ho[\N]{0}{S(n)}$ is also empty. This turns out to be circular, being equivalent to characterizing the morphisms of our unadulterated natural 
numbers. 

We can do slightly better. Our naive implementation of the natural numbers is not actually the correct notion of the "HoTT/MLTT naturals"
in sHoTT. Since HoTT has semantics in simplicial sets, we cannot naively restate definitions from simplicial sets in our bisimplicial setting.
Instead, we must rephrase definitions to account for the \textit{discrete} or \textit{constant} embeddings, which take simplicial sets to bisimplicial sets:
\[ \mathrm{disc} \colon \sSet \xrightarrow{-\square \Delta^0} \ssSet \qquad \qquad \mathrm{const} \colon \sSet \xrightarrow{\Delta^0\square-} \ssSet\]
So, we are presented two options for the natural numbers. When interpretting them in sHoTT, we can either interpret them as a discrete 
space with no simplicial data:

\begin{definition}
    \noindent Let $\Nd$ be the \textit{discrete} type generated by the constructors
\begin{itemize}
   \item $0 : \Nd$
   \item For all $n : \Nd$, a term $\su(n) : \Nd$.
\end{itemize}
The induction principle for $\Nd$ states that for any point-wise discrete type family $C : \Nd \to \U$ along with
\begin{itemize}
   \item A point $c_0 : C(0)$
   \item A function $c_s : \prd{n:\Nd}C(n) \to C(\su(n))$
\end{itemize}
There is a function $f : \prd{n:\Nd}C(n)$ such that $f(0) \equiv c_0$ and for all $n : \Nd$, $f(\su(n)) \equiv c_s(n, f(n))$.
\end{definition}
Or we can interpret the natural numbers as a Segal space with no discrete data:
\begin{definition}
    \noindent Let $\Ns$ be the \textit{Segal} type generated by the constructors
    \begin{itemize}
       \item $0 : \Ns$
       \item For all $n : \Ns$, a term $\su(n) : \Ns$.
    \end{itemize}
    The induction principle for $\Ns$ states that for any inner type family $C : \Ns \to \U$ along with
    \begin{itemize}
       \item A point $c_0 : C(0)$
       \item A function $c_s : \prd{n:\Ns}C(n) \to C(\su(n))$
    \end{itemize}
    There is a function $f : \prd{n:\Ns}C(n)$ such that $f(0) \equiv c_0$ and for all $n : \Ns$, $f(\su(n)) \equiv c_s(n, f(n))$.    
\end{definition}

Now, we know $\Nd$ is Segal because discrete spaces are Segal (lemma~\ref{lem:discretetypesseg}). On the other hand, we can show that
$\Ns$ is discrete:
\begin{lemma}
    $\Ns$ is discrete.
\end{lemma}
\begin{proof}
    Given that $\Ns$ is Segal, we can characterize the morphism space in a way that looks
    much more similar to book HoTT. Define a map $$\co' : \Ns \to\Ns \to\uc$$
    as
    \begin{align*}
        &\co'\,0\,0\defeq \Un&\\
        &\co'\,0\,S(n) \defeq \vo&\\
        &\co'\,S(n)\,0\defeq \vo&\\
        &\co'\,S(n)\,S(m)\defeq \co' \,n\,m&
    \end{align*}
    There is a dependent function $ r : \prd{n : \Ns} \co' \,n\,n$ defined as 
    \begin{align*}
        r\,0 \defeq \star \\
        r\,S(n) \defeq r(n)
    \end{align*}
    We can define $\eco' : \prd{n : \Ns}{m : \Ns} \ho[\Ns]{n}{m} \to \co'\,n\,m$ as
    \begin{align*}
        \eco'\,n\defeq \text{yon}^{\lam{m}\co'\,n\,m}_{n}(r(n))
    \end{align*}
    and $\dco' : \prd{n:\Ns}{m:\Ns}\co'\,n\,m \to \ho[\Ns]{n}{m}$ as
    \begin{align*}
        &\dco\,0\,0\,\_ \defeq \idhom{0}&\\
        &\dco\,S(n)\,S(m)\,u\defeq \lam{i}S((\dco\, n\, m\, u)i)&
    \end{align*}
    We must now prove that $\eco'$ and $\dco'$ are quasi-inverses. We do so by defining a map 
    $$\eta' : \prd{n:\Ns}{m:\Ns}{f : \ho[\Ns]{n}{m}}\dco'\,n\,m\,(\eco'\,n\,m f) =f$$
    as
    \begin{equation}
        \eta' \, n \defeq \text{yon}^{\lam{(m,f)}\dco'\,n\,m\,(\eco'\,n\,m f) =f}_n(\refl{\idhom{0}})
    \end{equation}
    with this being a valid definition as $\lam{(m,f)}\dco'\,n\,m\,(\eco'\,n\,m f) =f$ is covariant due to the
    covariance of $\lam{m}\ho[\Ns]{n}{m}$, which follows from the assumption that $\Ns$ is Segal.

    To define the final term showing that $\eco'$ and $\dco'$ are quasi-inverses, we note that we can perform a similar
    menueaver as in the non-Segal case and define a map showing that the fibers of $\co'$ are mere propositions. So, we define a map
     $\psi' : \prd{n : \Ns}{m : \Ns}\co\,n\,m \to (\co\,n\,m = \Un)$ as
    \begin{align*}
        &\psi'\,0\,0\,\_\defeq \refl{\Un}\\
        &\psi'\,S(n)\,S(m)\,u\defeq\psi\,n\,m\,u&
    \end{align*}
    and define 
$$\epsilon' : \prd{n:\Ns}{m:\Ns}{u : \co' \, n\, m}\eco'\,n\,m\,(\dco'\,n\,m \,u) =u$$
as
\begin{align*}
    \epsilon\,n\,m\,u \defeq \text{ap}_{((\psi'\,n\,m\,u)^{-1})_*}(\refl{\star})
\end{align*}
    Thus, for all $n,m : \Ns$ it is the case that
    $$\ho[\Ns]{n}{m}\simeq \co'\,n\,m.$$
    By \cite{program_homotopy_2013} theorem 2.13.1, though, we can further conclude that 
    $$m=_{\Ns} n \simeq \co'\,n\,m \simeq \ho[\Ns]{n}{m}.$$
    So, $\Ns$ is discrete.
\end{proof}

Since it is clear that $\Nd$ is Segal, and we have shown that $\Ns$ is discrete, we can craft maps
$f : \Nd \to \Ns$ and $g : \Ns \to \Nd$ by induction. The two maps simply "change" the type of a natural number, so they
form a quasi-inverse. That is to say, $\Ns$ and $\Nd$ are equivalent. Since $\Nd$ has the more restrictive elimination principle,
we opt to use this as our canonical natural numbers object. 

\end{document}