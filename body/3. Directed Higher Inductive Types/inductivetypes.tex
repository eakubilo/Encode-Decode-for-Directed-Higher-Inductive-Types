\documentclass[main.tex]{subfiles}

\begin{document}
\todo{define set, characterize path space of nd, show it is a set.}
As a warmup to more complicated directed higher inductive types, in this section we'll study the degenerate case of a dHIT: inductive types. More specifically, we study the natural numbers. A naive statement of the naturals in sHOTT is as follows:
\begin{definition}
    Let $\N$ be the type generated by the constructors
    \begin{itemize}
       \item $0 : \N$
       \item For all $n : \N$, a term $\su(n) : \N$.
    \end{itemize}
    The induction principle for $\N$ states that for any type family $C : \N \to \U$ along with
    \begin{itemize}
       \item A point $c_0 : C(0)$
       \item A function $c_s : \prd{n:\N}C(n) \to C(\su(n))$
    \end{itemize}
   There is a function $f : \prd{n:\N}C(n)$ such that $f(0) \equiv c_0$ and for all $n : \N$, $f(\su(n)) \equiv c_s(n, f(n))$.
   
\end{definition}

While this incarnation of the natural numbers seems syntactically permissible, its behavior is poor. For example, we can try to 
characterize the morphism type between two arbitrary natural numbers. To do so, we define a map $\co : \N \to \N \to \U$ with the 
defining equations 
\begin{align*}
    &\co\,0\,0\defeq \Un&\\
    &\co\,0\,S(n) \defeq \vo&\\
    &\co\,S(n)\,0\defeq \vo&\\
    &\co\,S(n)\,S(m)\defeq \co \,n\,m&
\end{align*}
We immediately run into issues while trying to define $$\eco : \prd{n : \N}{m : \N} \ho[\N]{n}{m} \to \co\,n\,m.$$ To do so, we'd have to supply
a term for $\eco\,0\,S(n)$ and $\eco\,S(n)\,0$. Evaluating the type of such a term, we find that we are required to come up with a map
that takes a morphism of type $\ho[\N]{0}{S(n)}$ to a term of type $\co\,0\,S(n)\equiv \vo$. Since we cannot inhabit $\vo$, we need to show 
that $\ho[\N]{0}{S(n)}$ is also empty. That is, we would need to be able to characterizations the morphisms of $\N$ to define $\eco$. Such a requirement is circular, thus untenable.

The issue lies in our naive implementation of the natural numbers. Since we have positively described $\N$ with no simplicial or homotopical data (morphisms or paths), it is reasonable to think that the type \textit{should} be Segal or discrete. Of course, if we could characterize the morphisms in the naturals then a proof of Segal or discrete structure would be straightforward. As we'll see in the next section, all directed higher inductive types have the structure of a Segal type. Thus, a workaround for us is to include the Segal constructur to the natural numbers definition and observe the consequences.

\begin{definition}
    \noindent Let $\Ns$ be the \textit{Segal} type generated by the constructors
    \begin{itemize}
       \item $0 : \Ns$
       \item For all $n : \Ns$, a term $\su(n) : \Ns$.
    \end{itemize}
    The induction principle for $\Ns$ states that for any inner type family $C : \Ns \to \U$ along with
    \begin{itemize}
       \item A point $c_0 : C(0)$
       \item A function $c_s : \prd{n:\Ns}C(n) \to C(\su(n))$
    \end{itemize}
    There is a function $f : \prd{n:\Ns}C(n)$ such that $f(0) \equiv c_0$ and for all $n : \Ns$, $f(\su(n)) \equiv c_s(n, f(n))$.    
\end{definition}
The elimination principles ensure that we can only eliminate into Segal types. We also define a natural numbers object that carries the discrete constructor, with elimination rules ensuring that it can only eliminate into discrete types:

\begin{definition}
    \noindent Let $\Nd$ be the \textit{discrete} type generated by the constructors
\begin{itemize}
   \item $0 : \Nd$
   \item For all $n : \Nd$, a term $\su(n) : \Nd$.
\end{itemize}
The induction principle for $\Nd$ states that for any point-wise discrete type family $C : \Nd \to \U$ along with
\begin{itemize}
   \item A point $c_0 : C(0)$
   \item A function $c_s : \prd{n:\Nd}C(n) \to C(\su(n))$
\end{itemize}
There is a function $f : \prd{n:\Nd}C(n)$ such that $f(0) \equiv c_0$ and for all $n : \Nd$, $f(\su(n)) \equiv c_s(n, f(n))$.
\end{definition}

We observe that $\Nd$ is Segal because discrete spaces are Segal (lemma~\cref{lem:discretetypesseg}). On the other hand, we can show that
$\Ns$ is discrete:
\begin{lemma}
    $\Ns$ is discrete.
\end{lemma}
\begin{proof}
    Given that $\Ns$ is Segal, we can characterize the morphism space in a way that looks similar to book HoTT characterization of the path space of $\mathbb{N}$. Define a map $$\co : \Ns \to\Ns \to\uc$$
    as
    \begin{align*}
        &\co\,0\,0\defeq \Unn&\\
        &\co\,0\,S(n) \defeq \voo&\\
        &\co\,S(n)\,0\defeq \voo&\\
        &\co\,S(n)\,S(m)\defeq \co \,n\,m&
    \end{align*}
    There is a dependent function $ r : \prd{n : \Ns} \El(\co \,n\,n)$ defined as 
    \begin{align*}
        r\,0 \defeq \star \\
        r\,S(n) \defeq r(n)
    \end{align*}
    We can define $\eco : \prd{n : \Ns}{m : \Ns} \ho[\Ns]{n}{m} \to \El(\co\,n\,m)$ as
    \begin{align*}
        \eco\,n\defeq \text{yon}^{\lam{m}\El(\co\,n\,m)}_{n}(r(n))
    \end{align*}
    and $\dco : \prd{n:\Ns}{m:\Ns}\El(\co\,n\,m) \to \ho[\Ns]{n}{m}$ as
    \begin{align*}
        &\dco\,0\,0\,\_ \defeq \idhom{0}&\\
        &\dco\,S(n)\,S(m)\,u\defeq \lam{i}S((\dco\, n\, m\, u)i)&
    \end{align*}
    We must now prove that $\eco$ and $\dco$ are quasi-inverses. We do so by defining a map 
    $$\eta : \prd{n:\Ns}{m:\Ns}{f : \ho[\Ns]{n}{m}}\dco\,n\,m\,(\eco\,n\,m f) =f$$
    as
    \begin{equation}
        \eta \, n \defeq \text{yon}^{\lam{(m,f)}\dco\,n\,m\,(\eco\,n\,m f) =f}_n(\refl{\idhom{0}})
    \end{equation}
    with this being a valid definition as $\lam{(m,f)}\dco\,n\,m\,(\eco\,n\,m f) =f$ is covariant due to the
    covariance of $\lam{m}\ho[\Ns]{n}{m}$, which follows from the assumption that $\Ns$ is Segal.

    To define the final term showing that $\eco$ and $\dco$ are quasi-inverses, we define a map showing that the fibers of $\El(\co)$ are mere propositions. To do so, we define the following map
     $\psi : \prd{n , m: \Ns}{u,v : \El(\co\,n\,m)}u=_{\El(\co\,n\,m)} v$ as
    \begin{align*}
        &\psi\,0\,0\,\star\,\star\defeq \refl{\star}\\
        &\psi\,S(n)\,S(m)\,u\,v\defeq\psi(\,n\,m\,u\,v)&
    \end{align*}
    and define 
$$\epsilon : \prd{n:\Ns}{m:\Ns}{u : \El(\co \, n\, m)}\eco\,n\,m\,(\dco\,n\,m \,u) =u$$
as
\begin{align*}
    \epsilon\,n\,m\,u \defeq \psi(n\,m\,(\eco\,n\,m\,(\dco\,n\,m \,u))\,u)
\end{align*}
    Thus, for all $n,m : \Ns$ it is the case that
    $$\ho[\Ns]{n}{m}\simeq \El(\co\,n\,m).$$

    We can similarly use $\El(\co)$ to characterize the path space of $\Ns$ as well. Since this is a routine calculation that follows very similarly to the one we see above, we will only highlight the differences. The maps $\eco$ and $\eta$ will follow similarly, except they will use transport as opposed to the Yoneda lemma. On the other hand, $\dco$ and $\epsilon$ will follow identitally. From that description, we can further conclude that 
    $$m=_{\Ns} n \simeq \El(\co\,n\,m) \simeq \ho[\Ns]{n}{m}.$$
    So, $\Ns$ is discrete.
\end{proof}

Since it is clear that $\Nd$ is Segal, and we have shown that $\Ns$ is discrete, we can craft maps
$f : \Nd \to \Ns$ and $g : \Ns \to \Nd$ by induction. Both maps will essentially be the identity function. Thus, it is clear that $\Ns$ and $\Nd$ are equivalent. Since $\Nd$ has the more restrictive elimination principle,
we opt to use this as our canonical natural numbers object. 

\end{document}